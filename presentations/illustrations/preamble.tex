\usepackage{ifthen}
\usepackage{pgf}
\usepackage{ulem}
\usepackage{tikz}
\usetikzlibrary{backgrounds}
\usetikzlibrary{calc}
\usetikzlibrary{positioning}
\usetikzlibrary{patterns}
\usetikzlibrary{decorations.pathreplacing}
\usetikzlibrary{arrows,shapes}
\usepackage{tikzpeople}
\usepackage{amssymb}

% Active adversary
\newcommand{\actadv}[4][]{
  \ifthenelse{\equal{#1}{}}{
    \draw[fill=white] #2 rectangle #3 node[pos=.5] {};
    \draw ($#2 + (3,1.5)$) node[devil,mirrored,minimum size=1cm] {};
    \draw[fill=white, fill opacity=0.5] #2 rectangle #3 node[pos=.5] {};
    \draw #2 rectangle #3 node[pos=.5] {\footnotesize #4};
  }{
    \onlyenv<#1>
    \draw[fill=white] #2 rectangle #3 node[pos=.5] {};
    \draw ($#2 + (3,1.5)$) node[devil,mirrored,minimum size=1cm] {};
    \draw[fill=white, fill opacity=0.5] #2 rectangle #3 node[pos=.5] {};
    \draw #2 rectangle #3 node[pos=.5] {\footnotesize #4};
    \endonlyenv
  }
}

% Inactive adversary
\newcommand{\inactadv}[4][]{
  \ifthenelse{\equal{#1}{}}{
    \draw[fill=white] #2 rectangle #3 node[pos=.5] {};
    \draw ($#2 + (3,1.5)$) node[devil,mirrored,minimum size=1cm] {};
    \draw[fill=white, fill opacity=0.4] #2 rectangle #3 node[pos=.5] {};
    \draw[fill=gray, fill opacity=0.5] #2 rectangle #3 node[pos=.5] {};
    \draw #2 rectangle #3 node[pos=.5] {\footnotesize #4};
  }{
    \onlyenv<#1>
    \draw[fill=white] #2 rectangle #3 node[pos=.5] {};
    \draw ($#2 + (3,1.5)$) node[devil,mirrored,minimum size=1cm] {};
    \draw[fill=white, fill opacity=0.4] #2 rectangle #3 node[pos=.5] {};
    \draw[fill=gray, fill opacity=0.5] #2 rectangle #3 node[pos=.5] {};
    \draw #2 rectangle #3 node[pos=.5] {\footnotesize #4};
    \endonlyenv
  }
}

% Active stack frame
\newcommand{\actsf}[4][]{
  \ifthenelse{\equal{#1}{}}{
    \draw[fill=white] #2 rectangle #3 node[pos=.5] {\footnotesize #4};
  }{
   \draw<#1>[fill=white] #2 rectangle #3 node[pos=.5] {#4};
 }
}

% Inactive stack frame
\newcommand{\inactsf}[4][]{
  \ifthenelse{\equal{#1}{}}{
    \draw[fill=white] #2 rectangle #3 node[pos=.5] {};
    \draw[fill=gray, fill opacity=0.5] #2 rectangle #3 node[pos=.5] {};
    \draw #2 rectangle #3 node[pos=.5] {\footnotesize #4};
  }{
    \draw<#1>[fill=white] #2 rectangle #3 node[pos=.5] {};
    \draw<#1>[fill=gray, fill opacity=0.5] #2 rectangle #3 node[pos=.5] {};
    \draw<#1> #2 rectangle #3 node[pos=.5] {\footnotesize #4};
  }
}

\newcommand{\capbrace}[3][sp1]{
  \draw [decorate,decoration={brace,amplitude=10pt,mirror,raise=4pt},yshift=0pt]
  #2 -- #3 node[draw=black] (#1) [black,midway,xshift=0.8cm] {};

}

\newcommand{\stdstackstart}{
  \scope
    \clip (-.1,-.1) rectangle (6.1,15.1);
    \fill[fill=white] (0,0) rectangle (6,15);
    \draw (0,0) -- (0,15);
    \draw (6,0) -- (6,15);
    \draw[fill=gray!50] (0,-.5) rectangle (6,2.5) node[pos=.5,color=black] {\footnotesize higher stack frames...};
  \endscope
  % \draw[->] (-2,0) -- node[midway,sloped,above] {stack grows upward} (-2,15);
  \draw (-1.5,0) node {};
  \draw (-1.5,15) node {};
}


% stolen from https://tex.stackexchange.com/questions/14225/is-there-the-easiest-way-to-toggle-show-hide-navigational-grids-in-tikz/14230
\makeatletter
\newif\if@showgrid@grid
\newif\if@showgrid@left
\newif\if@showgrid@right
\newif\if@showgrid@below
\newif\if@showgrid@above
\tikzset{%
    every show grid/.style={},
    show grid/.style={execute at end picture={\@showgrid{grid=true,#1}}},%
    show grid/.default={true},
    show grid/.cd,
    labels/.style={font={\sffamily\small},help lines},
    xlabels/.style={},
    ylabels/.style={},
    keep bb/.code={\useasboundingbox (current bounding box.south west) rectangle (current bounding box.north west);},
    true/.style={left,below},
    false/.style={left=false,right=false,above=false,below=false,grid=false},
    none/.style={left=false,right=false,above=false,below=false},
    all/.style={left=true,right=true,above=true,below=true},
    grid/.is if=@showgrid@grid,
    left/.is if=@showgrid@left,
    right/.is if=@showgrid@right,
    below/.is if=@showgrid@below,
    above/.is if=@showgrid@above,
    false,
}

\def\@showgrid#1{%
    \begin{scope}[every show grid,show grid/.cd,#1]
    \if@showgrid@grid
    \begin{pgfonlayer}{background}
    \draw [help lines]
        (current bounding box.south west) grid
        (current bounding box.north east);
%
    \pgfpointxy{1}{1}%
    \edef\xs{\the\pgf@x}%
    \edef\ys{\the\pgf@y}%
    \pgfpointanchor{current bounding box}{south west}
    \edef\xa{\the\pgf@x}%
    \edef\ya{\the\pgf@y}%
    \pgfpointanchor{current bounding box}{north east}
    \edef\xb{\the\pgf@x}%
    \edef\yb{\the\pgf@y}%
    \pgfmathtruncatemacro\xbeg{ceil(\xa/\xs)}
    \pgfmathtruncatemacro\xend{floor(\xb/\xs)}
    \if@showgrid@below
    \foreach \X in {\xbeg,...,\xend} {
        \node [below,show grid/labels,show grid/xlabels] at (\X,\ya) {\X};
    }
    \fi
    \if@showgrid@above
    \foreach \X in {\xbeg,...,\xend} {
        \node [above,show grid/labels,show grid/xlabels] at (\X,\yb) {\X};
    }
    \fi
    \pgfmathtruncatemacro\ybeg{ceil(\ya/\ys)}
    \pgfmathtruncatemacro\yend{floor(\yb/\ys)}
    \if@showgrid@left
    \foreach \Y in {\ybeg,...,\yend} {
        \node [left,show grid/labels,show grid/ylabels] at (\xa,\Y) {\Y};
    }
    \fi
    \if@showgrid@right
    \foreach \Y in {\ybeg,...,\yend} {
        \node [right,show grid/labels,show grid/ylabels] at (\xb,\Y) {\Y};
    }
    \fi
    \end{pgfonlayer}
    \fi
    \end{scope}
}
\makeatother
