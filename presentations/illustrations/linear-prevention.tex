\begin{frame}
  \frametitle{Linear Stack Capabilities}
  \begin{tikzpicture}[scale=.5, every node={scale=.5}]
    % recurrent parts
    \stdstackstart

    % Caller 
    \actsf{(0,2.5)}{(6,5)}{caller stack frame} 
    \begin{onlyenv}<+>
      \capbrace{(6.5,2.5)}{(6.5,16)}
      \draw node[right=0cm of sp1] { stack pointer };
    \end{onlyenv}


    % Split the stack pointer
    \begin{onlyenv}<+>
      \capbrace[sp2]{(6.5,2.5)}{(6.5,5)}
      \capbrace[usp]{(6.5,5)}{(6.5,16)}
      \draw node[right=0cm of usp] { stack pointer };
      \draw node[right=0cm of sp2] { private stack pointer };
    \end{onlyenv}


    % Callee called
    \begin{onlyenv}<+>
      \capbrace[sp2]{(6.5,2.5)}{(6.5,5)}
      \capbrace[usp]{(6.5,5)}{(6.5,16)}
      \draw node[right=0cm of usp] { stack pointer };
      \draw node[fill=gray!50, right=0cm of sp2] { return data pointer };
      \inactsf{(0,2.5)}{(6,5)}{caller stack frame} 
      \actsf{(0,5)}{(6,6.5)}{callee stack frame}
    \end{onlyenv}

    % Return from callee
    \begin{onlyenv}<+>
      \capbrace[sp2]{(6.5,2.5)}{(6.5,5)}
      \capbrace[usp]{(6.5,5)}{(6.5,16)}
      \draw node[right=0cm of usp] { stack pointer };
      \draw node[right=0cm of sp2] { private stack pointer };
    \end{onlyenv}

    % Caller splices the stack frames
    \begin{onlyenv}<+>
      \capbrace{(6.5,2.5)}{(6.5,16)}
      \draw node[right=0cm of sp1] { stack pointer };
    \end{onlyenv}

  \end{tikzpicture}
\end{frame}

\begin{frame}
    \frametitle{Linear Capabilities Prevent Attack 1}
  \begin{tikzpicture}[scale=.5, every node={scale=.5}]
    % recurrent parts
    \stdstackstart
    \draw (16.5,5.5) node {register file};
    \begin{scope}
      \clip (15.9,1.1) rectangle (17.1,5.1);
      \foreach \x in {1,2,...,4}
      {
        \draw[fill=white] (16,\x) rectangle (17,\x+1) node[pos=.5,color=teal] {};
      };
    \end{scope}
    \node (r1) at (16.75,4.5) {};
    \node (r2) at (16.75,3.6) {};

    % Caller 
    \actsf{(0,2.5)}{(6,5)}{caller stack frame} 
    \begin{onlyenv}<+>
      \capbrace[sp1]{(6.5,2.5)}{(6.5,16)}
      \draw node[right=0cm of sp1] { stack pointer };
      \draw (r1) edge[->,red,thick,bend left] (sp1);
    \end{onlyenv}


    % Split the stack pointer
    \begin{onlyenv}<+>
      \capbrace[sp2]{(6.5,2.5)}{(6.5,5)}
      \draw (r2) edge[->,red,thick,bend left] (sp2);
      \draw [decorate,decoration={brace,aspect=0.2,amplitude=10pt,mirror,raise=4pt},yshift=0pt]
      (6.5,5) -- (6.5,16) node[draw=black] (usp) [black,pos=0.2,xshift=0.8cm] {};
      \draw (r1) edge[->,red,thick,bend left] (usp);
      \draw node[right=0cm of usp] { stack pointer };
      \draw node[right=0cm of sp2] { private stack pointer };
    \end{onlyenv}


    % Callee called
    \begin{onlyenv}<+>
      \capbrace[sp2]{(6.5,2.5)}{(6.5,5)}
      \draw [decorate,decoration={brace,aspect=0.2,amplitude=10pt,mirror,raise=4pt},yshift=0pt]
      (6.5,5) -- (6.5,16) node[draw=black] (usp) [black,pos=0.2,xshift=0.8cm] {};
      \draw (r1) edge[->,red,bend left] (usp);
      \draw node[right=0cm of usp] { stack pointer };

      \draw node[fill=gray,opacity=0.5, right=0cm of sp2] { \phantom{return data pointer} };
      \draw node[right=0cm of sp2] { return data pointer };
      \inactsf{(0,2.5)}{(6,5)}{caller stack frame} 
      \actadv{(0,5)}{(6,8)}{callee stack frame}
    \end{onlyenv}
    \draw<.-> (r2) edge[->,red,bend left] (sp2);


    \begin{onlyenv}<.->
      \begin{scope}
        \fill[gray!20] (14,9) rectangle (18,14);
        \draw (14,9) -- (14,14);
        \draw (18,9) -- (18,14);
      \end{scope}
    \end{onlyenv}
    \draw<.>[fill=white] (14,10) rectangle node[color=red] {} (18,10.5);

    % Callee stores stack pointer
    \begin{onlyenv}<+->
      \draw [decorate,decoration={brace,aspect=0.2,amplitude=10pt,mirror,raise=4pt},yshift=0pt]
      (8,5) -- (8,16) node[draw=black] (usp) [black,pos=0.2,xshift=0.8cm] {};
      \draw (16,15) node {heap memory};
    \end{onlyenv}

    \begin{onlyenv}<.>
      \draw (14,10.25) edge[->,red,thick,bend right] (usp);
      \draw[fill=white] (14,10) rectangle node[color=red] {\scriptsize old sp} (18,10.5);
      \draw node[right=0cm of usp] { stack pointer };
      \inactsf{(0,2.5)}{(6,5)}{caller stack frame} 
      \actadv{(0,5)}{(6,8)}{callee stack frame}
      \capbrace[sp2]{(6.5,2.5)}{(6.5,5)}
      \draw node[fill=gray,opacity=0.5, right=0cm of sp2] { \phantom{return data pointer} };
      \draw node[right=0cm of sp2] { return data pointer };
    \end{onlyenv}

    % Return from callee
    \begin{onlyenv}<+->
      \draw (14,10.25) edge[->,red,bend right] (usp);
      \draw[fill=gray!50] (14,10) rectangle node[color=red] {\scriptsize old sp} (18,10.5);
      \actsf{(0,2.5)}{(6,5)}{caller stack frame} 
      \capbrace[sp2]{(6.5,2.5)}{(6.5,5)}
      \draw node[right=0cm of sp2] { private stack pointer };
      \draw node[right=0cm of usp] { stack pointer };
    \end{onlyenv}

    \draw<+>[teal,very thick] (6.4,4.4) -- (7.4,5.4);
    \draw<.>[teal,very thick] (7.4,4.4) -- (6.4,5.4);
    \draw<.> node[teal,cloud,cloud puffs=10.8,cloud puff arc=110, aspect=3, draw, fill=white, align=center] () at (14,4) { Splice fails,\\ stack pointer is linear!};  
  \end{tikzpicture}  
\end{frame}

\begin{frame}
  \frametitle{Linear Capabilities Prevent Attack 2}
  \begin{tikzpicture}[scale=.5, every node={scale=.5}]
    % recurrent parts
    \stdstackstart

    \draw (16.5,5.5) node {register file};
    \begin{scope}
      \clip (15.9,1.1) rectangle (17.1,5.1);
      \foreach \x in {1,2,...,4}
      {
        \draw[fill=white] (16,\x) rectangle (17,\x+1) node[pos=.5,color=teal] {};
      };
    \end{scope}
    \node (r1) at (16.75,4.5) {};
    \node (r2) at (16.75,3.6) {};

    % Caller 
    \actsf{(0,2.5)}{(6,5)}{caller stack frame} 
    \begin{onlyenv}<+>
      \capbrace{(6.5,2.5)}{(6.5,16)}
      \draw node[right=0cm of sp1] { stack pointer };
      \draw (r1) edge[->,red,thick,bend left] (sp1);
    \end{onlyenv}


    % Split the stack pointer
    \begin{onlyenv}<+>
      \capbrace[sp2]{(6.5,2.5)}{(6.5,5)}
      \capbrace[usp]{(6.5,5)}{(6.5,16)}
      \draw (r2) edge[->,red,thick,bend left] (sp2);
      \draw (r1) edge[->,red,thick,bend left] (usp);
      \draw node[right=0cm of usp] { stack pointer };
      \draw node[right=0cm of sp2] { private stack pointer };
    \end{onlyenv}


    % Callee called
    \begin{onlyenv}<+>
      \capbrace[usp]{(6.5,5)}{(6.5,16)}
      \draw (r1) edge[->,red,bend left] (usp);
      \draw node[right=0cm of usp] { stack pointer };
      \inactsf{(0,2.5)}{(6,5)}{caller stack frame} 
      \actadv{(0,5)}{(6,8)}{callee stack frame}
    \end{onlyenv}

    \begin{onlyenv}<.->
      \capbrace[sp2]{(6.5,2.5)}{(6.5,5)}
      \draw node[fill=gray!50,right=0cm of sp2] { \phantom{return data pointer} };
      \draw (r2) edge[->,red,bend left] (sp2);
      \draw node[right=0cm of sp2] { return data pointer };
    \end{onlyenv}

    \begin{onlyenv}<+>

      \capbrace[usp]{(13.5,5)}{(13.5,16)}
      \draw (6,13.25) edge[red,thick,->,in=45] (usp);
      \draw node[right=0cm of usp] { stack pointer };

      \draw[fill=white] (0,13) rectangle (6,13.5) node[pos=.5,color=red] {\footnotesize sp};
      \inactsf{(0,2.5)}{(6,5)}{caller stack frame} 
      \actadv{(0,5)}{(6,8)}{callee stack frame}
    \end{onlyenv}

    % At this point we have a linear point stored in a part of memory that it governs. In other words, no other capabilities can grant access to this part of memory, to the capabilitiy is essentially lost.
%    \draw<+> node[teal,cloud,cloud puffs=10.8,cloud puff arc=110, aspect=3, draw, fill=white, align=center] () at (4,10) { Stack pointer\\essentially lost.}; 

    \begin{onlyenv}<+->

      \capbrace[usp]{(13.5,5)}{(13.5,16)}
      \draw (6,13.25) edge[red,thick,->,in=45] (usp);
      \draw node[right=0cm of usp] { stack pointer };

      \begin{scope}
        \fill[fill=gray!50] (0,2.5) rectangle (6,15) node[pos=.5,color=black] {};
        \draw (0,0) -- (0,15);
        \draw (6,0) -- (6,15);
      \end{scope}
      \draw[fill=gray!50] (0,13) rectangle (6,13.5) node[pos=.5,color=red] {\footnotesize sp};
      \inactsf{(0,2.5)}{(6,5)}{caller stack frame} 
      \inactadv{(0,5)}{(6,8)}{callee stack frame}
      \actsf{(0,8)}{(6,10.5)}{callee (2) stack frame} 
  \end{onlyenv}

  \draw<+>[teal,very thick] (-0.5,7.5) -- (6.5,11);
  \draw<.>[teal,very thick] (-0.5,11) -- (6.5,7.5);
  \draw<.> node[teal,cloud,cloud puffs=10.8,cloud puff arc=110, aspect=3, draw, fill=white, align=center] () at (14,7) { No stack pointer\\for callee (2)!};  

  \end{tikzpicture}
\end{frame}


\begin{frame}
  \frametitle{Linear Capabilities Prevent Attack 3}
  \begin{tikzpicture}[scale=.5, every node={scale=.5}]
    % recurrent parts
  \scope
    \clip (-.1,-.1) rectangle (6.1,15.1);
    \fill[fill=white] (0,0) rectangle (6,15);
    \draw (0,0) -- (0,15);
    \draw (6,0) -- (6,15);
    \draw[fill=gray!50] (0,-.5) rectangle (6,1.5) node[pos=.5,color=black] {\footnotesize higher stack frames...};
  \endscope
  % \draw[->] (-2,0) -- node[midway,sloped,above] {stack grows upward} (-2,15);
  
  \begin{scope}
    \fill[gray!50] (13,0) rectangle (18,13);
    \draw (13,0) -- (13,13);
    \draw (18,0) -- (18,13);
    \draw (15.5,14) node {code memory};
  \end{scope}

\note{Caller calls callee (split stack pointer to make ret-ptr-d)}
  \begin{onlyenv}<+->
    \actsf{(0,1.5)}{(6,4.5)}{caller stack frame}
    \draw[fill=white] (13,1) rectangle (18,3) node[pos=.5] {\footnotesize caller code};
  \end{onlyenv}
  \begin{onlyenv}<.>
    \capbrace[sp]{(6.1,1.5)}{(6.1,15)}
    \draw node[right=0cm of sp] { sp };
  \end{onlyenv}

  \begin{onlyenv}<+->
    \capbrace[ret1]{(6.1,1.5)}{(6.1,4.5)}
    \draw node[fill=gray!50,right=0cm of ret1] { \phantom{ret-ptr-d} };
    \draw node[right=0cm of ret1] { ret-ptr-d };
  \end{onlyenv}
  \begin{onlyenv}<.->
    \draw [decorate,decoration={brace,aspect=0.3,amplitude=4pt,raise=4pt},yshift=0pt]
    (12.9,1) -- (12.9,3) node[draw=black] (ret1c) [black,pos=0.3,xshift=-0.6cm] {};
    \draw node[fill=gray!50,left=0cm of ret1c] { \phantom{ret-ptr-c} };
    \draw node[left=0cm of ret1c] { ret-ptr-c };
  \end{onlyenv}
 
  \begin{onlyenv}<.-.(3)>
    \capbrace[sp]{(6.1,4.5)}{(6.1,15)}
    \draw node[right=0cm of sp] { sp };
  \end{onlyenv}



  \begin{onlyenv}<+-+(3)>
    \draw[fill=gray!30] (13,1) rectangle (18,3) node[pos=.5] {\footnotesize caller code};
    \draw[fill=white] (13,4) rectangle (18,6) node[pos=.5] {\footnotesize callee code};
    \actadv{(0,4.5)}{(6,7.5)}{callee stack frame}
    \inactsf{(0,1.5)}{(6,4.5)}{caller stack frame}
  \end{onlyenv}

\note{Callee saves ret-ptr-d on the heap.}
  \begin{onlyenv}<+->
    \draw (13,4.75) edge[red,bend right,thick,->] (ret1);
  \end{onlyenv}
\note{Callee saves ret-ptr-c on the heap.}
  \begin{onlyenv}<+->
    \draw (13,4.75) edge[red,bend right,thick,->] (ret1c);
  \end{onlyenv}


\note{Callee calls callee (1)}
\note{Before the callee jumps, it is assumed that it has checked the global "max height" of its stack pointer!}
  \begin{onlyenv}<+-+(4)>
    \capbrace[ret2]{(6.1,4.5)}{(6.1,7.5)}
    \draw node[fill=gray!50,right=0cm of ret2] { \phantom{ret-ptr-d (1)} };
    \draw node[right=0cm of ret2] { ret-ptr-d (1) };
  \end{onlyenv}
  \begin{onlyenv}<.-.(1)>
    \capbrace[sp]{(6.1,7.5)}{(6.1,15)}
    \draw node[right=0cm of sp] { sp };
  \end{onlyenv}

  \begin{onlyenv}<+-+(1)>
    \inactsf{(0,1.5)}{(6,4.5)}{caller stack frame}
    \inactadv{(0,4.5)}{(6,7.5)}{callee stack frame}
    \actsf{(0,7.5)}{(6,10.5)}{callee stack frame (1)}
    \draw[fill=gray!30] (13,1) rectangle (18,3) node[pos=.5] {\footnotesize caller code};
    \draw[fill=gray!30] (13,4) rectangle (18,6) node[pos=.5] {\footnotesize callee code};
    \draw[fill=white] (13,7) rectangle (18,9) node[pos=.5] {\footnotesize callee code (1)};
  \end{onlyenv}

\note{Callee (1) calls callee (2)}
  \begin{onlyenv}<+-+(2)>
    \capbrace[ret3]{(6.1,7.5)}{(6.1,10.5)}
    \draw node[fill=gray!50,right=0cm of ret3] { \phantom{ret-ptr-d (2)} };
    \draw node[right=0cm of ret3] { ret-ptr-d (2) };
  \end{onlyenv}
  \begin{onlyenv}<.->
    \capbrace[sp]{(6.1,10.5)}{(6.1,15)}
    \draw node[right=0cm of sp] { sp };
  \end{onlyenv}

  \begin{onlyenv}<+->
    \inactsf{(0,1.5)}{(6,4.5)}{caller stack frame}
    \inactadv{(0,4.5)}{(6,7.5)}{callee stack frame}
    \inactsf{(0,7.5)}{(6,10.5)}{callee stack frame (1)}
  \end{onlyenv}
  
  \begin{onlyenv}<.->
    \draw[fill=gray!30] (13,1) rectangle (18,3) node[pos=.5] {\footnotesize caller code};
    \draw[fill=gray!30] (13,4) rectangle (18,6) node[pos=.5] {\footnotesize callee code};
    \draw[fill=gray!30] (13,7) rectangle (18,9) node[pos=.5] {\footnotesize callee code (1)};
    \draw[fill=white] (13,10) rectangle (18,12) node[pos=.5] {\footnotesize callee code (2)};
  \end{onlyenv}
  \begin{onlyenv}<.-.(1)>
    \actadv{(0,10.5)}{(6,13.5)}{callee stack frame (2)}
  \end{onlyenv}

\note{Callee (2) has a capability for Callee and can thus access the stored ret-ptr-d}
  \begin{onlyenv}<+->
    \draw [decorate,decoration={brace,amplitude=4pt,mirror,raise=4pt},yshift=0pt]
    (18.1,4) -- (18.1,6) node[draw=black] (calleecap) [black,midway,xshift=0.6cm] {};
    \draw (18,11) edge[red,bend left,thick,->] (calleecap);
  \end{onlyenv}

\note{Callee (2) uses ret-ptr-d to return}
  \begin{onlyenv}<+->
    \actsf{(0,1.5)}{(6,4.5)}{caller stack frame}
    \draw node[fill=white,right=0cm of ret1] { ret-ptr-d };
    \draw node[fill=white,left=0cm of ret1c] { ret-ptr-c };
    \draw (13,4.75) edge[red,bend right,thick,->] (ret1);
  \end{onlyenv}

  \begin{onlyenv}<+->
    \draw[teal,very thick] (6.4,4.4) -- (7.4,5.4);
    \draw[teal,very thick] (7.4,4.4) -- (6.4,5.4);
    \draw node[teal,cloud,cloud puffs=10.8,cloud puff arc=110, aspect=3, draw, fill=white, align=center] () at (10,6) { Splice fails!};  
  \end{onlyenv}
  
\note{Splicing: fails because ends of capabilities don't meet. callee stack frame (1) cannot be empty (it will at least contain 42).}
  % \begin{onlyenv}<+->
  %   \draw node[teal,cloud,cloud puffs=10.8,cloud puff arc=110, aspect=3, draw, fill=white, align=center] () at (8,7.5) {Caller tries to splice sp \\and ret-ptr-d, but fails!};  
  % \end{onlyenv}
\note{Before the caller makes another call/return, it will check that the stack height matches the global max stack height.}
\end{tikzpicture}

\end{frame}

 	
%%% Local Variables:
%%% TeX-master: "presentation-illustrations"
%%% End: