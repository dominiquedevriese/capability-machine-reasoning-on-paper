\documentclass[compsoc,conference,letterpaper,fleqn]{IEEEtran}
 
\usepackage{cite}
\usepackage{amsmath,amsfonts,amssymb,amsthm}
\usepackage{thmtools}
\usepackage{stmaryrd}
\usepackage{natbib}
\usepackage{url}
\usepackage{array}
\usepackage{arydshln}
\usepackage{ifthen}
\usepackage{ifpdf}
\usepackage{verbatim}
\usepackage{mathpartir}
\usepackage{listings}
\usepackage{hyperref}
\lstset{
  basicstyle=\ttfamily,
  columns=fullflexible,
  keepspaces=true,
  mathescape
}

\usepackage{rotating}

\usepackage{mathtools}
\DeclarePairedDelimiter\ceil{\lceil}{\rceil}
\DeclarePairedDelimiter\floor{\lfloor}{\rfloor}

\usepackage{enumitem}
\newlist{enumproof}{enumerate}{10}
\setlist[enumproof]{label*=\arabic*.}

\usepackage{tikz}
\usepackage{multirow,bigdelim}

% %make latex preview-mode work with natbib...
% \usepackage[displaymath,floats,graphics,textmath,footnotes]{preview}

\newtheorem{theorem}{Theorem}

\newcommand{\forcenewline}{$\phantom{v}$\\}
\newcommand{\judgment}[2]{\paragraph{#1}\hspace{\stretch{1}}\fbox{$#2$}}

\newcommand{\update}[2]{[#1 \mapsto #2]}
\newcommand{\sem}[1]{\left\llbracket #1 \right\rrbracket}

% Math notation
\newcommand{\restrictfun}[1]{|_{#1}}
\newcommand{\parfun}{\rightharpoonup}
\newcommand{\finparfun}{\xrightharpoonup{\textit{\tiny{fin}}}}
\newcommand{\monnefun}{\xrightarrow{\textit{\tiny{mon, ne}}}}
\newcommand{\monfun}{\xrightarrow{\textit{\tiny{mon}}}}
\newcommand{\nefun}{\xrightarrow{\textit{\tiny{ne}}}}
\newcommand{\fun}{\rightarrow}
\newcommand{\defeq}{\stackrel{\textit{\tiny{def}}}{=}}
\newcommand{\nequal}[1][n]{\stackrel{\tiny{#1}}{=}}
\renewcommand{\nsim}[1][n]{\stackrel{\tiny{#1}}{\simeq}}

\newcommand\subsetsim{\mathrel{\ooalign{\raise.2ex\hbox{$\subset$}\cr
      \hidewidth\lower.8ex\hbox{\scalebox{0.9}{$\sim$}}\hidewidth\cr}}}
\newcommand\supsetsim{\mathrel{\ooalign{\raise.2ex\hbox{$\supset$}\cr
      \hidewidth\lower.8ex\hbox{\scalebox{0.9}{$\sim$}}\hidewidth\cr}}}
\newcommand{\nsubsim}[1][n]{\stackrel{\tiny{#1}}{\subsetsim}}
\newcommand{\nsupsim}[1][n]{\stackrel{\tiny{#1}}{\supsetsim}}

\newcommand{\nsubeq}[1][n]{\stackrel{\tiny{#1}}{\subseteq}}
\newcommand{\nsupeq}[1][n]{\stackrel{\tiny{#1}}{\supseteq}}

\newcommand{\union}{\mathbin{\cup}}
\DeclareMathOperator{\dom}{dom}
\newcommand{\blater}{\mathop{\blacktriangleright}}
\newcommand{\id}{\var{id}}
\newcommand{\undefined}{\mathit{undefined}}

\newcommand{\powerset}[1]{\mathcal{P}(#1)}

\newcommand{\false}{\mathit{false}}
\newcommand{\true}{\mathit{true}}


% cofes
\newcommand{\cofe}{c.o.f.e.}
\newcommand{\cofes}{\cofe{}'s}
\newcommand{\CatC}{\mathbb{C}}
\newcommand{\CatP}{\mathbb{P}}

% Comments
\newcommand\lau[1]{{\color{purple} \sf \footnotesize {LS: #1}}\\}
\newcommand\dominique[1]{{\color{purple} \sf \footnotesize {DD: #1}}\\}
\newcommand\lars[1]{{\color{purple} \sf \footnotesize {LB: #1}}\\}

% Variables
\newcommand{\var}[1]{\mathit{#1}}
\newcommand{\hs}{\var{ms}}
\newcommand{\ms}{\hs}
\newcommand{\hv}{\var{hv}}
\newcommand{\rv}{\var{rv}}
\newcommand{\lv}{\var{lv}}
\newcommand{\gl}{\var{g}}
\newcommand{\pc}{\mathit{pc}}
\newcommand{\pcreg}{\mathrm{pc}}
\newcommand{\addr}{\var{a}}
\newcommand{\offset}{\var{offset}}
\newcommand{\word}{\var{w}}
\newcommand{\start}{\var{base}}
\newcommand{\addrend}{\var{end}}
\newcommand{\pwlv}{\var{pwl}}
\newcommand{\mem}{\var{mem}}
\newcommand{\reg}{\var{reg}}
\newcommand{\heapseg}{\var{ms}}
\newcommand{\heap}{\var{mem}}
\newcommand{\mode}{\var{mode}}
\newcommand{\perm}{\var{perm}}
\newcommand{\permp}{\var{permPair}}
\newcommand{\roll}{\var{roll}}
\newcommand{\instr}{\var{instr}}
\newcommand{\stdcap}[1][(\perm,\gl)]{\left(#1,\start,\addrend,\addr \right)}
\newcommand{\adv}{\var{adv}}
\newcommand{\msframe}{ms_\var{frame}}
\newcommand{\link}{\var{link}}
\newcommand{\stk}{\var{stk}}
\newcommand{\flag}{\var{flag}}
\newcommand{\nwl}{\var{nwl}}
\newcommand{\pwl}{\var{pwl}}
\newcommand{\sta}{\var{sta}}
\newcommand{\cnst}{\var{cnst}}
\newcommand{\olf}{\var{offsetLinkFlag}}
\newcommand{\prp}{\var{prp}}
\newcommand{\env}{\var{env}}
\newcommand{\cls}{\var{cls}}
\newcommand{\unused}{\var{unused}}
\newcommand{\act}{\var{act}}


% Memory projections
\newcommand{\plainproj}[1]{\mathrm{#1}}
\newcommand{\memheap}[1][\Phi]{#1.\plainproj{mem}}
\newcommand{\memreg}[1][\Phi]{#1.\plainproj{reg}}

\newcommand{\updateHeap}[3][\Phi]{#1\update{\plainproj{mem}.#2}{#3}}
\newcommand{\updateReg}[3][\Phi]{#1\update{\plainproj{reg}.#2}{#3}}

% Configuration end states
\newcommand{\failed}{\textsl{failed}}
\newcommand{\halted}{\textsl{halted}}

% Functions
\newcommand{\plainfun}[2]{
  \ifthenelse{\equal{#2}{}}
  {\mathit{#1}}
  {\mathit{#1}(#2)}
}
\newcommand{\decode}{\plainfun{decode}{}}
\newcommand{\encode}{\plainfun{encode}{}}
\newcommand{\encodePerm}{\mathit{encodePerm}}
\newcommand{\encodePermPair}{\plainfun{encodePermPair}{}}
\newcommand{\encodeLoc}{\mathit{encodeLoc}{}}
\newcommand{\decodePermPair}{\plainfun{decodePermPair}}
\newcommand{\decodePerm}[1]{\plainfun{decodePerm}{#1}}
\newcommand{\updatePcPerm}[1]{\plainfun{updatePcPerm}{#1}}

\newcommand{\executeAllowed}[1]{\plainfun{executeAllowed}{#1}}
\newcommand{\nonZero}[1]{\plainfun{nonZero}{#1}}
\newcommand{\readAllowed}[1]{\plainfun{readAllowed}{#1}}
\newcommand{\writeAllowed}[1]{\plainfun{writeAllowed}{#1}}
\newcommand{\withinBounds}[1]{\plainfun{withinBounds}{#1}}
\newcommand{\stdUpdatePc}[1]{\plainfun{updatePc}{#1}}

\newcommand{\readCond}[1]{\plainfun{readCondition}{#1}}
\newcommand{\writeCond}[1]{\plainfun{writeCondition}{#1}}
\newcommand{\execCond}[1]{\plainfun{executeCondition}{#1}}
\newcommand{\entryCond}[1]{\plainfun{enterCondition}{#1}}

\newcommand{\revokeTemp}[1]{\plainfun{revokeTemp}{#1}}
\newcommand{\erase}[2]{\floor*{#1}_{\{#2\}}}
\newcommand{\activeReg}[1]{\plainfun{active}{#1}}

% World operations
\newcommand{\future}{\mathbin{\sqsupseteq}}
\newcommand{\pub}{\var{pub}}
\newcommand{\priv}{\var{priv}}
\newcommand{\futurewk}{\mathbin{\sqsupseteq}^{\var{pub}}}
\newcommand{\futurestr}{\mathbin{\sqsupseteq}^{\var{priv}}}
\newcommand{\heapSat}[3][\heap]{#1 :_{#2} #3}
\newcommand{\memSat}[3][n]{\heapSat[#2]{#1}{#3}}
\newcommand{\memSatPar}[4][n]{\heapSat[#2]{#1 , #4}{#3}}

\newcommand{\monwknefun}{\xrightarrow[\text{\tiny{$\futurewk$}}]{\textit{\tiny{mon, ne}}}}
\newcommand{\monstrnefun}{\xrightarrow[\text{\tiny{$\futurestr$}}]{\textit{\tiny{mon, ne}}}}


% Assembly labels
\newcommand{\codelabel}[1]{\mathit{#1}}
\newcommand{\init}{\codelabel{init}}
\newcommand{\malloc}{\codelabel{malloc}}
\newcommand{\counter}{\codelabel{counter}}
\newcommand{\iocap}{\codelabel{iocap}}

% Type(s)
\newcommand{\type}[1]{\mathrm{#1}}
\newcommand{\asmType}{\plaindom{AsmType}}


% Domains
\newcommand{\plaindom}[1]{\mathrm{#1}}
\newcommand{\Caps}{\plaindom{Cap}}
\newcommand{\Words}{\plaindom{Word}}
\newcommand{\Addrs}{\plaindom{Addr}}
\newcommand{\ExecConfs}{\plaindom{ExecConf}}
\newcommand{\RegName}{\plaindom{RegisterName}}
\newcommand{\Regs}{\plaindom{Reg}}
\newcommand{\Heaps}{\plaindom{Mem}}
\newcommand{\Mems}{\Heaps}
%\newcommand{\HeapSegments}{\plaindom{MemSegment}}
\newcommand{\HeapSegments}{\plaindom{MemSeg}}
\newcommand{\MemSegments}{\HeapSegments}
\newcommand{\Confs}{\plaindom{Conf}}
\newcommand{\Instrs}{\plaindom{Instructions}}
\newcommand{\nats}{\mathbb{N}}
\newcommand{\ints}{\mathbb{Z}}
\newcommand{\Perms}{\plaindom{Perm}}
\newcommand{\Globals}{\plaindom{Global}}

\newcommand{\Rel}{\plaindom{Rel}}
\newcommand{\Rels}{\plaindom{Rels}}
\newcommand{\States}{\plaindom{State}}
\newcommand{\RegionNames}{\plaindom{RegionName}}
\newcommand{\Regions}{\plaindom{Region}}
\newcommand{\Reg}{\plaindom{Reg}}
\newcommand{\Worlds}{\plaindom{World}}
\newcommand{\Wor}{\plaindom{Wor}}
\newcommand{\Worwk}{\Wor_{\futurewk}}
\newcommand{\Worstr}{\Wor_{\futurestr}}
\newcommand{\xiwk}{\xi_{\var{wk}}}
\newcommand{\xistr}{\xi_{\var{str}}}
\newcommand{\StorePred}{\plaindom{MemSegPred}}
\newcommand{\UPred}[1]{\plaindom{UPred}(#1)}
\newcommand{\DCPred}[1]{\plaindom{P}^\downarrow(#1)}

\newcommand{\Views}{\plaindom{View}}

% LR
\newcommand{\intr}[2]{\mathcal{#1}}
\newcommand{\valueintr}[1]{\intr{V}{#1}}
\newcommand{\exprintr}[1]{\intr{E}{#1}}
\newcommand{\contintr}[1]{\intr{K}{#1}}
\newcommand{\regintr}[1]{\intr{R}{#1}}
\newcommand{\stdvr}{\valueintr{\asmType}}
\newcommand{\stder}{\exprintr{\asmType}}
\newcommand{\stdrr}{\regintr{\asmType}}
\newcommand{\stdkr}{\contintr{\asmType}}
\newcommand{\observations}{\mathcal{O}}
\newcommand{\npair}[2][n]{\left(#1,#2 \right)}

% Reference register/memory
\newcommand{\refreg}[1]{\lfloor #1 \rfloor}
\newcommand{\refheap}[1]{\langle #1 \rangle_m}

% Instructions
% No arguments
\newcommand{\zinstr}[1]{\mathtt{#1}}
\newcommand{\fail}{\zinstr{fail}}
\newcommand{\halt}{\zinstr{halt}}
% One argument
\newcommand{\oneinstr}[2]{\zinstr{#1} \; #2}
\newcommand{\jmp}[1]{\oneinstr{jmp}{#1}}
% Two arguments
\newcommand{\twoinstr}[3]{\zinstr{#1} \; #2 \; #3}
\newcommand{\restricttwo}[2]{\twoinstr{restrict}{#1}{#2}}
\newcommand{\jnz}[2]{\twoinstr{jnz}{#1}{#2}}
\newcommand{\isptr}[2]{\twoinstr{isptr}{#1}{#2}}
\newcommand{\geta}[2]{\twoinstr{geta}{#1}{#2}}
\newcommand{\getb}[2]{\twoinstr{getb}{#1}{#2}}
\newcommand{\gete}[2]{\twoinstr{gete}{#1}{#2}}
\newcommand{\getp}[2]{\twoinstr{getp}{#1}{#2}}
\newcommand{\getl}[2]{\twoinstr{getl}{#1}{#2}}
\newcommand{\move}[2]{\twoinstr{move}{#1}{#2}}
\newcommand{\store}[2]{\twoinstr{store}{#1}{#2}}
\newcommand{\load}[2]{\twoinstr{load}{#1}{#2}}
\newcommand{\lea}[2]{\twoinstr{lea}{#1}{#2}}
% Three arguments
\newcommand{\threeinstr}[4]{\zinstr{#1} \; #2 \; #3 \; #4}
\newcommand{\restrict}[3]{\threeinstr{restrict}{#1}{#2}{#3}}
\newcommand{\subseg}[3]{\threeinstr{subseg}{#1}{#2}{#3}}
\newcommand{\plus}[3]{\threeinstr{plus}{#1}{#2}{#3}}

% Permissions
\newcommand{\plainperm}[1]{\mathrm{#1}}
\newcommand{\noperm}{\plainperm{o}}
\newcommand{\readonly}{\plainperm{ro}}
\newcommand{\readwrite}{\plainperm{rw}}
\newcommand{\exec}{\plainperm{rx}}
\newcommand{\entry}{\plainperm{e}}
\newcommand{\rwx}{\plainperm{rwx}}
% PWL permissions
\newcommand{\readwritel}{\plainperm{rwl}}
\newcommand{\rwl}{\readwritel}
\newcommand{\rwlx}{\plainperm{rwlx}}

% Global/local
\newcommand{\local}{\plainperm{local}}
\newcommand{\glob}{\plainperm{global}}

\newcommand{\localityReg}{\var{localityReg}}
\newcommand{\localReg}{\var{localReg}}
\newcommand{\globalReg}{\var{globalReg}}

% Views
\newcommand{\plainview}[1]{\mathrm{#1}}
\newcommand{\perma}{\plainview{perm}}
\newcommand{\temp}{\plainview{temp}}
\newcommand{\revoked}{\plainview{revoked}}

% OP sem
\newcommand{\diverge}[1][n]{\not\Downarrow_{#1}}
\newcommand{\step}[1][]{\rightarrow_{#1}}

% Conv defs
\newcommand{\lookingat}[3]{\ensuremath{#1} \text{ is looking at } \ensuremath{#2} \text{ followed by } \ensuremath{#3}}
\newcommand{\pointstostack}[3]{\ensuremath{#1} \text{ points to stack with } \ensuremath{#2} \text{ used and } \ensuremath{#3} \text{ unused}}
\newcommand{\nonlocal}[1]{\ensuremath{#1} \text{ is non-local}}

% Macros
\newcommand{\scall}[3]{\mathtt{scall} \; #1([#2],[#3])}





\usepackage[colorinlistoftodos,prependcaption,textsize=tiny]{todonotes}


% domi: IEEEtran.cls sets font size to 10bp, apparently.
% not sure what the difference is with 10pt.
\DeclareMathSizes{10bp}{8}{7}{6}

%\def\IEEEbibitemsep{0pt}

\makeatletter
\setlength\mpr@andskip{1em}
\def\mpr@lineskip{5em}
\def\MathparLineskip{\mpr@lesslineskip}
\makeatother

\newlength{\oldmathindent}
\newenvironment{withmathindent}[1]{\setlength{\oldmathindent}{\mathindent}\setlength{\mathindent}{#1}}{\setlength{\mathindent}{\oldmathindent}}



\begin{document}
\setlength{\mathindent}{.2cm}

\title{Reasoning about a Capability Machine with Local Capabilities\\
 Provably Safe Stack and Return Pointer Management (without OS Support).}


\author{%
\IEEEauthorblockN{Lau~Skorstengaard}
\IEEEauthorblockA{Aarhus University, Denmark\\
Email: lau@cs.au.dk} \and
\IEEEauthorblockN{Dominique~Devriese}
  \IEEEauthorblockA{iMinds-DistriNet, KU Leuven, Belgium\\
Email: dominique.devriese@cs.kuleuven.be} \and
\IEEEauthorblockN{Lars~Birkedal}\IEEEauthorblockA{Aarhus University, Denmark\\
Email: birkedal@cs.au.dk}}

\maketitle

\begin{abstract}
  abc
\end{abstract}

\section*{Terminology ?}

\emph{capability safe}: 
Our unary logical relation is a semantic \emph{definition}
of which values for the
pc / register-file / word are capability safe. 

\emph{adversary}: It doesn't sound ``sexy'' to constantly say ``an untrusted
piece of code'' should we adopt the convention that adversary refers
to an unstrusted piece of code?

\section{Introduction}
\label{sec:introduction}
\todo[inline]{See the latex comments for the next paragraphs!}
% Possible high-level intros:
% 0) Arms race compilers
%
% 1) Security guarantees vs obstacles / arms race
\begin{comment}
  Security on processors is today subject to an arms race between
  processor manufacturers and hackers. On the hacker side, a constant
  effort is put into exploiting current processor designs and
  circumventing the security measures set in place to prevent known
  exploits. On the processor manufacturer side, new security measures
  are put in place to prevent new exploits. The arms race would come
  to a grinding halt if the security meassures provided more than
  obstacles and actually provided low-level security guarantees. One
  kind of low-level machines that provide low-level security is a
  capability machine.
\end{comment}
% 2) Fully abstract compilation
High-level languages provide guarantees such as encapsulation of local
state and well-bracketedness of function calls. These guarantees are
often relied upon when reasoning about the correctness and security of
a program. High-level programs are compiled to machine code, so in
order to rely on the correctness and security guarantees provided by
the high-level language, it needs to be proven that the translation
preserves these properties. Translated programs often interact with
machine code that is not translated from the same high-level language,
so in order to preserve the guarantees from the high-level languages,
the low-level machine needs to provide some kind of security
guarantees. In other words, a target of secure compilation needs to
provide some security guarantees. Modern processors provide many
security meassures, but very little in terms of guarantees. A machine
that provides some means of security guarantees is a capability machine.

% 3) Memory access-control
% I don't like this angle and can't find a good way to sell it.

% 4) History
% Proposals throughout history.
% Lack of formal account of properties-guarantees.

%%% a)
% Current low-level protection coarse grained memory protection or
% properties of high-level languages are not really enforced?

% Capability machines offer fine-grained memory protection
% Too technical
A capability machine is a low-level machine that offers fine-grained
memory protection.
% Capability machines and unforgeable tokens for memory access. 
This is done by introducing unforgeable tokens to
the machine. These tokens are called capabillities and grant a kind of
authority and a range of authority. 
% Dynamic checks encapsulation, formal model

 We will refer to the capabilities
that grant a combination of read, write, or execute permission as memory
capabilities. 
 Capabilities with the necessary permissions must be
presented every time an instruction that manipulates the memory is
executed.

Memory capabilities provide memory protection, but they do not provide
a way to set up security domains. When you jump to an untrusted piece
of code that is supposed to jump back to you, then you have to either
revoke all your capabilities or pass to the piece of code you do not
trust. This short coming can be overcome by adding something like the
enter capability from the M-Machine\todo{add reference}. The enter
capability is an opaque capability which can only be used for a
jump. When jumped to it grants read and execute permission which
allows the code that is now executing to read capabilities stored in
memory. 
The CHERI processor's ccall achieves a similar property\todo{Add reference}.

Capabilities are irrevocable, so when we pass a capability has to
an untrusted program, then we have to assume that the program keeps it
around indefenitely. This means that we cannot reuse the piece of
memory that the capability governs essentially creating a memory
leak. The CHERI processor has a special kind of capabilities called
\emph{local capabilities}\todo{reference?}\ that introduces a simple
kind of temporal information control which allows for a simple kind of
capability revocation. This is achieved by adding a tag to every
capability that marks it as either local or global. Local capabilities
can only be written through a capability with a new permission called
\emph{permit write local}.
% More about local capabilities.

With just memory capabilities, enter capabailities, and local
capabilities, the capability machine is very powerful and can
enforce properties that high-level languages promise. In most
high-level languages, we expect local state to be private and
encapsulated. The memory capabilities make sure that local state
cannot be accessed without a capability to do so. With the enter
capabaility, we can regain control of local state after passing
control to an untrusted piece of code which gives us encapsulation
when control is passed to an untrusted program.

Many high-level languages guarantee that function function calls are
well-bracketed. This can easily be achieved by having a trusted call
stack. On a capability machine, it is possible to enforce well
barcketedness without a trusted stack. When ensuring well-bracketed
the two main challenges calls are 1) preventing storage of return
pointers and 2) encapsulation of the local stack frame. If we don't
have the first property, then an adversary can store the return
capability in a call and jump to it in a later call. If the local
stack frame is not properly encapsulated, then an adversary can break
the well-bracketedness by jumping to a different calls return
capability.

In order to deal with the two challenges, we make sure that all return
capabilities are local and that there are no global capabilities with
permit write local permission. This makes sure that there is no way to
save a return capability in persistent local state.\lau{this does not really make
sense as the local stack frame can be seen as local state} It is,
however, to limiting to not provide any means to store the return
capabilities as a piece of code may need to store a return capability
before jumping to an adversary. It is therefore needed to provide a
local capability with permit write local permission which can be used
as a stack. The stack should now be the only place one can store the
return capability, but it should also be the only place one can store
the stack capability. 

Using local return capabilities and putting a stack abstraction on a
local capability permit write local capability is a good start, but as
it turns out, it is not wnough. An adversary could do the following 
\begin{enumerate}
\item In a first invocation, fill the stack frame with copies of the
  current stack capability.
\item In a later call, the adversary may be able to load the stack
  capability from the previous call.
\item If we have used a larger part of the stack than in the first
  call, then the adversary has access to part of our local stack frame.
\end{enumerate}

The above attack shows that it is tricky to ensure
well-bracketedness. It raises the question whether it is possible to
do so it is completely watertight. The attack utilized the stack, so
it is necessary to have a trusted stack or can we do without? In
general it raises the question, how do we reason about capability
machines and local capabilities, so we are sure that what ever calling
scheme we come up with actually provably ensures well-bracketed calls.

In the following paper, we present the following contributions
\begin{enumerate}
\item 
\end{enumerate}
 

%%% b)




% Examples of capability machines?


% Enter capabilities

% Local capabilities

% High-level languages often provide guarantees such as encapsulation of private state and well-bracketedness of calls - often not clear how well this is enforced. Just enforced when interacting with other programs written in the same high-level language or is it also guaranteed when interacting with assembly programs.

% High-level programs compiled to assembly can ensure these properties.

% 

\subsection{Introduce capability machines}
\subsection{Enforcing properties on a capability machine}

Privacy of local state using memory capabilities.
Encapsulation of private state using enter pointers or CHERI's ccall.
(mention in passing secure compilation, full abstraction)

\subsection{Well-bracketed control flow and encapsulation of stack frames}

Hard to enforce using standard capabilities:
\begin{itemize}
\item Well-bracketed control flow can be broken if the adversary stores return pointers.
(alternative: trusted stack)
\item Encapsulation of local stack frames can be broken if the adversary stores stack pointers.
\end{itemize}

\subsection{Local capabilities}

\begin{itemize}
\item Introduced by CHERI for this purpose.
\item Explain how they can be used for passing return pointers and stack pointers.
\item But this is not watertight.
\item Example attack:
\begin{itemize}
\item In a first invocation, adversary fills the entire unused part of the stack with copies of the current stack pointer
\item In a second invocation, adversary can load the old stack pointer of the stack
\item He now has access to private state that was pushed on the stack in between.
\end{itemize}
\item So:
\begin{itemize}
\item Can we make it (provably) watertight?
\item Is the trusted stack necessary? Why not just pass return pointers locally?
\item General question: How to reason about capability machines and
local capabilities?
\end{itemize}
\end{itemize}

\subsection{Contributions}

\begin{itemize}
\item Formal model of a simple but representative capability machine with local capabilities
\item Detailed study of how to do safe stack and return pointer management in this setting, taking into account:
\begin{itemize}
\item untrusted adversary
\item higher-order code: callbacks passed to and received from the adversary
\item efficient stack management
\item no OS support
\end{itemize}
\item Logical relation for reasoning about code in this capability machine
\item Fundamental theorem that expresses the guarantees offered by a capability machine for untrusted code
\item Technical contributions: reuse existing ideas in a new way:
\begin{itemize}
\item replace biorthogonal closure by a single orthogonal closure
because assembly languages remove the distinction between the
continuation and the arguments
\end{itemize}
\end{itemize}
LB: I think we should have some discussion of the flexibility /
  strength of our LR.  The LR defines which computations we think of
  as well-behaved.   
 We should give simple / trivial examples of
  code not in the LR. But we may also want to emphasize that the LR is 
  not too restricting, e.g., it does not enforce a certain calling
  scheme (examplied by Example \verb!f1! which
  does not use the stack and the following examples which do use the
  stack). This ties in to the discussion of biorthogonality, see the
  following paragraphs. 

LB: we need to relate this carefully to Hur-Dreyer, who used
      biorthogonality even though they also worked with an assembly
      language. (If I recall correctly, they assumed some properties
      of the low level language, which is why the biortho was the
      right ting, but we need to check.)

LS: It is my impression that they were interested in the relation
between high-level and low-level programs. They are therefore in
particular interested in low-level programs compiled from high-level
programs which means that the continuation is always invoked in the 
same way. Our realisation was that anything we pass as an argument 
in a "call" can be used as a continuation, so the continuation 
relation was redundant. Even though we do try to give our programs
some structure using the macros, our logical relation is strong enough
to handle unstructured programs as well.
\begin{itemize}
\item STSs with public/private transitions for dealing with local capabilities: play the same role as before, but in different places of the LR.
\end{itemize}
\begin{itemize}
\item Demonstrate all of this on several examples:
\begin{itemize}
\item security examples
\item the most challenging examples from existing literature on reasoning about well-bracketed control flow in lambda calculi
\item some compartmentalisation result?
\item whatever else we do..
\end{itemize}
\end{itemize}

\subsection{Outline}

\section{A capability machine with local capabilities}

Formalisation

Explain our model of linking

\section{Stack and return pointer management using local capabilities}

\begin{itemize}
\item Issues to consider:
\begin{itemize}
\item no global store-local capabilities
\item clearing the stack
\item clearing (some) registers
\item constructing return pointers
\item constructing callbacks
\item invoking the adversary's callbacks
\end{itemize}
\item Reusable macro-instructions for stack and return pointer management
\begin{itemize}
\item taking care of the stuff above
\end{itemize}
\end{itemize}

LB: maybe we can present the examples informally before the logical relation and then return to them after the logical relation (to present the formal properties) ?

\section{Logical relation}
In order to be able to reason about programs running on the capability
machine and prove properties such as well-bracketedness and
encapsulation about them, we define a unary logical relation.

% Intuition
The logical relation is a semantic model of all the words on the
capability machine that are definitely safe to give to an
adversary. Whether a capability is safe to give to an adversary
depends heavily on the contents of the memory since instructions as well
as capabilities may reside in memory. The memory will be modeled using
worlds. A world consists of a finite set of regions, and each region
can be understood as a protocol describing the possible evolution of
part of the memory. Memory changes during an
execution since new memory can be allocated and part of the existing
memory can change. We use two future world relations to capture
such changes to the memory in our model.
%

\subsection{Regions}
%Regions intuition
We have three types of regions: $\perma$, $\temp$, and $\revoked$. The
$\perma$ and $\temp$ regions contains a state transition system with
public and private transitions as known from logical relations
from high-level languages\todo{reference}. The state transition
systems can be understood as protocols the memory must follow. The
protocols imposed on memory by $\perma$ regions stay in place
indefinitely. This means that any capability, $\local$ or $\glob$, can
depend on these protocols. The protocols imposed on memory by $\temp$
regions can be revoked at certain points during execution
corresponding to the points when $\local$ capabilities can be
revoked. This means that $\local$ capabilties safely can depend on the
protocols imposed on memory by $\temp$ regions, but $\glob$
capabilities cannot as the $\glob$ capability may stay around
throughout the execution of a program, but the $\temp$ regions may not
stay around. 

For monotonicity reasons, a region cannot disappear, but
we need a way to remove $\temp$ regions. This is where the $\revoked$ region comes
into the picture. The $\revoked$ region contains no state transition
system and puts no requirements on the memory it models. It serves
simply as a mask for the $\temp$ regions that have been revoked.



\begin{theorem}\label{thm:world-existence}
  There exists a \cofe{} $\Wor$ and preorders $\futurestr$ and
  $\futurewk$ such that $(\Wor,\futurestr)$ and $(\Wor,\futurewk)$ are
  preordered \cofes{}, and there exists an isomorphism $\xi$ such that
  {\small
    \begin{align*}              
      \xi : \Wor & \cong \blater (\nats \finparfun \Regions)\\
      \Regions & = \{\revoked\} \uplus \\
                 &\hskip -2cm \{\temp\} \times \States \times \Rels \times (\States \fun (\Wor \monwknefun \UPred{\HeapSegments})) \uplus \\
                 & \hskip -2cm \{\perma\} \times \States \times \Rels \times (\States \fun (\Wor \monstrnefun \UPred{\HeapSegments}))
    \end{align*}
  } and for $W, W' \in \Wor$
  \begin{align*}
    W' \futurestr W & \Leftrightarrow \xi(W') \futurestr \xi(W)   \\
    W' \futurewk W & \Leftrightarrow \xi(W') \futurewk \xi(W)
  \end{align*}
\end{theorem}

\subsection{Worlds}
For this logical relation, we use step-indexed Kripke worlds. They are
defined by the $\Regions$ given in 




%Let $\iota.v$ be the projection of the view of a region.
Worlds are finite partial maps from region names to regions,
and region names are simply natural numbers
\begin{align*}
  \Worlds & = \RegionNames \finparfun \Regions\\
 \RegionNames & = \nats.
\end{align*}





\subsubsection{Future worlds}
There are two future world relations a \emph{public future world} relation and a \emph{private future world} relation. In both future world relations, we have extension ordering, so a future world $W'$ of $W$ has at least the same regions as $W$, but it may add new regions imposing new protocols on part of the memory. The two future world relations differ in the way existing regions are allowed to evolve. In a \emph{public future region} 
\begin{mathpar}
  \inferrule{ \dom(W') \supseteq \dom(W)\\ 
    \forall r \in \dom(W) \ldotp W'(r) \futurewk W(r) }
  { W' \futurewk W }
  \and
  \inferrule{ \dom(W') \supseteq \dom(W)\\ 
    \forall r \in \dom(W) \ldotp W'(r) \futurestr W(r) }
  { W' \futurestr W }
\end{mathpar}


{\bf Public future regions}
The \emph{public future} region relations satisfies the following properties:
\begin{mathpar}
  \inferrule{  (s,s') \in \phi_\pub \\
    (v,\phi_\pub,\phi,H) = (v',\phi_\pub',\phi',H')}
  {  (v',s',\phi_\pub',\phi',H') \futurewk (v,s,\phi_\pub,\phi,H) }
  \and
  \inferrule{ (\temp,s,\phi_\pub,\phi,H) \in \Regions }
  { (\temp,s,\phi_\pub,\phi,H) \futurewk \revoked }
  \and
  \inferrule{ }
  { \revoked \futurewk \revoked }
\end{mathpar}

{\bf Private future regions}
The \emph{private future} region relation satisfies the following properties:
\begin{mathpar}
  \inferrule{  (s,s') \in \phi \\
    (v,\phi_\pub,\phi,H) = (v',\phi_\pub',\phi',H')}
  {  (v',s',\phi_\pub',\phi',H') \futurestr (v,s,\phi_\pub,\phi,H) }
  \and
  \inferrule{ r \in \Regions }
  { r \futurestr (\temp,s,\phi_\pub,\phi,H) }
  \and
  \inferrule{ r \in \Regions }
  { r \futurestr \revoked }
\end{mathpar}


The two future world relations satisfy the following properties: They allow for any extension of the current world and all existing worlds are allowed to move to an appropriate future region. That is

\subsection{Logical relation}


\subsection{Fundamental theorem of logical relations}

\subsection{Discussion/comparison with related Dreyer-Neis-Birkedal and with Hur-Dreyer}


\section{Examples}
\begin{itemize}
\item Ticket dispenser
\item The awkward example and variants
\item A sandboxing example?

For example, an untrusted advertisement scenario with initialization code
that registers a redraw callback. The redraw callback gets temporary
read-write access to a framebuffer.

\item Some compartmentalisation result?
\end{itemize}

\section{Discussion, future work}
\begin{itemize}
\item A general well-bracketed control flow result?
\begin{itemize}
\item what would that result say?
\item possible idea: fully abstract compilation from an assembly language with
a trusted stack to one without
\item the LR and some of the lemmas already imply well-bracketed control flow, as seen in examples
\end{itemize}
\item Relation to local parameters in Scala, Algol/Pascal second-class function parameters?
\item Stack clearing realistic?
\item Non-modularity of heap allocation requirement for adversary callbacks
\end{itemize}

\section{Related work}

\begin{itemize}
\item Dreyer-Neis-Birkedal
\item CHERI papers
\item Akram's thesis
\item \url{http://2016.splashcon.org/event/splash-2016-oopsla-gentrification-gone-too-far-affordable-2nd-class-values-for-fun-and-co-effect}
\item other papers that enforce well-bracketed control flow at low level
(using a trusted stack manager)
\begin{itemize}
\item \url{http://ieeexplore.ieee.org/abstract/document/7536364/}
\item \url{http://ieeexplore.ieee.org/abstract/document/7536366/}
\item other stuff?
\end{itemize}
\end{itemize}

\section{Conclusion}

\section*{Acknowledgements}
\label{sec:acknowledgements}

This research was supported in part by the ModuRes Sapere Aude Advanced Grant from The Danish Council for Independent Research for the Natural Sciences (FNU).
Dominique Devriese holds a postdoctoral fellowship from the Research Foundation - Flanders (FWO).

\bibliographystyle{plainnat}
\bibliography{references}
\end{document}
\grid
