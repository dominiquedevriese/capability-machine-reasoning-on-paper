\documentclass{article}

\usepackage{amsmath}
\usepackage{amsthm}
\usepackage{amsfonts}
\usepackage{thmtools}
\usepackage{stmaryrd}
\usepackage{natbib}
\usepackage{url}

\declaretheorem[numbered=yes,name=Lemma]{lemma}
\declaretheorem[numbered=yes,name=Definition]{definition}

\newcommand{\update}[2]{[#1 \mapsto #2]}%TODO change this to nicer notation \Phi[heap.a \mapsto w]

\newcommand{\var}[1]{\mathit{#1}}
\newcommand{\hv}{hv}
\newcommand{\rv}{rv}
\newcommand{\lv}{lv}
\newcommand{\pc}{\mathit{pc}}
\newcommand{\pcreg}{\mathrm{pc}}
\newcommand{\addr}{a}
\newcommand{\word}{w}
\newcommand{\len}{len}
\newcommand{\mem}{mem}
\newcommand{\reg}{reg}
\newcommand{\heap}{heap}
\newcommand{\mode}{mode}
\newcommand{\perm}{perm}
\newcommand{\failed}{failed}
\newcommand{\halted}{halted}
\newcommand{\false}{\mathit{false}}
\newcommand{\true}{\mathit{true}}
\newcommand{\decode}{\mathit{decode}}
\newcommand{\updatePcPerm}{\mathit{updatePcPerm}}
\newcommand{\executeAllowed}{\mathit{executeAllowed}}
\newcommand{\nonZero}[1]{\mathit{nonZero}(#1)}
\newcommand{\readAllowed}[1]{\mathit{readAllowed}(#1)}
\newcommand{\writeAllowed}[1]{\mathit{writeAllowed}(#1)}

\newcommand{\plaindom}[1]{\mathrm{#1}}
\newcommand{\Caps}{\plaindom{Cap}}
\newcommand{\Words}{\plaindom{Word}}
\newcommand{\Addrs}{\plaindom{Addr}}
\newcommand{\Mems}{\plaindom{Mem}}
\newcommand{\RegName}{\plaindom{RegName}}
\newcommand{\Regs}{\plaindom{Reg}}
\newcommand{\Heaps}{\plaindom{Heap}}
\newcommand{\Confs}{\plaindom{Conf}}
\newcommand{\Lens}{\plaindom{Len}}
\newcommand{\Instrs}{\plaindom{Instructions}}
\newcommand{\nats}{\mathbb{N}}

\newcommand{\Perms}{\plaindom{Perm}}

\newcommand{\refreg}[1]{\lfloor #1 \rfloor}
\newcommand{\refheap}[1]{\langle #1 \rangle_h}

\newcommand{\instr}[1]{\mathtt{#1}}
\newcommand{\fail}{\instr{fail}}
\newcommand{\halt}{\instr{halt}}
\newcommand{\oneinstr}[2]{\instr{#1} \; #2}
\newcommand{\jmp}[1]{\oneinstr{jmp}{#1}}


\newcommand{\twoinstr}[3]{\instr{#1} \; #2 \; #3}
\newcommand{\jnz}[2]{\twoinstr{jnz}{#1}{#2}}
\newcommand{\isptr}[2]{\twoinstr{isptr}{#1}{#2}}
\newcommand{\setptr}[2]{\twoinstr{setptr}{#1}{#2}}
\newcommand{\move}[2]{\twoinstr{move}{#1}{#2}}
\newcommand{\store}[2]{\twoinstr{store}{#1}{#2}}
\newcommand{\load}[2]{\twoinstr{load}{#1}{#2}}
\newcommand{\lea}[2]{\twoinstr{lea}{#1}{#2}}

\newcommand{\threeinstr}[4]{\instr{#1} \; #2 \; #3 \; #4}
\newcommand{\restrict}[3]{\threeinstr{restrict}{#1}{#2}{#3}}
\newcommand{\subseg}[3]{\threeinstr{subseg}{#1}{#2}{#3}}
\newcommand{\plus}[3]{\threeinstr{plus}{#1}{#2}{#3}}

\newcommand{\plainperm}[1]{\mathrm{#1}}
\newcommand{\readonly}{\plainperm{ro}}
\newcommand{\readwrite}{\plainperm{rw}}
\newcommand{\exec}[1]{\plainperm{x_{#1}}}
\newcommand{\entry}[1]{\plainperm{e_{#1}}}
\newcommand{\key}{\plainperm{k}}

\newcommand{\sem}[1]{\left\llbracket #1 \right\rrbracket}

\begin{document}
$\RegName$ contains $\pcreg$, but is otherwise some undefined, finite set.
\begin{align*}
\Addrs &::= \nats & & &
\Words &::= \Caps + \nats \\
\Regs  &::= \RegName \rightarrow \Words & & &
\Heaps &::= \Addrs \rightarrow \Words \\
\Perms &::= \{\readonly, \readwrite, \exec{u}, \exec{p}, \entry{u}, \entry {p}, \key\} & & &
\Mems  &::= \Regs \times \Heaps \\
\Lens  &::= \nats & & &
\Caps  &::= \Perms \times \Lens \times \Addrs \\
\Confs &::= \Mems + \{\failed, \halted\}
\end{align*}
Notation:
$$\begin{array}{rcl}
i      &\in& \Instrs \\
r      &\in& \RegName\\
\mem   &::=& (\reg,\heap)\\
\len   &\in& \nats \\
\pc    &\in& \Caps \\
\pcreg &\in& \RegName \\
\Phi   &::=& \mem \in \Confs\\
\addr      &\in& \Addrs\\
\perm  &\in& \Perms\\
(\perm,len,\addr) &\in& \Caps \\
\end{array}$$
Further definitions:
$$\begin{array}{rcl}
\lv    &::=& \refreg{r} \\
\hv    &::=& \refheap{r}\\
\rv    &::=& n \mid \lv \\
i      &::=& \fail \mid \halt \mid 
             \jmp{\lv} \mid \jnz{\lv}{\rv} \mid
             \isptr{\lv}{\rv} \mid \setptr{\lv}{\rv} \mid \\
       &   & \lea{\lv}{\rv} \mid\move{\lv}{\rv} \mid \load{\lv}{\hv} \mid \store{\hv}{\rv} \mid  \\
       &   & \restrict{\lv}{\rv}{\rv} \mid \subseg{\lv}{\rv}{\rv} \mid \plus{\lv}{\rv}{\rv}
\end{array}$$

\begin{align*}
\decode &:\nats \rightarrow \Instrs
\end{align*}
%TODO for some decode function

\begin{align*}
\Phi & \rightarrow \sem{\decode(\Phi.\reg(\pcreg))}(\Phi) & & \text{if $\Phi.\reg(\pcreg) = (\exec{\_},{\_},{\_})$} \\
\Phi & \rightarrow \failed                                      & & \text{otherwise}
\end{align*}
\begin{align*}
  \executeAllowed(\perm) &=
                           \begin{cases}
                             \true & \text{if } \perm \in \{\exec{u},\exec{p},\entry{u},\entry{p}\} \\
                             \false & \text{otherwise}
                           \end{cases} \\
  \readAllowed{\perm} &=
                           \begin{cases}
                             \true & \text{if } \perm \in \{\readonly, \readwrite, \exec{u},\exec{p}\} \\
                             \false & \text{otherwise}
                           \end{cases} \\
  \writeAllowed{\perm} &=
                           \begin{cases}
                             \true & \text{if } \perm \in \{\readwrite\} \\
                             \false & \text{otherwise}
                           \end{cases} \\
  \updatePcPerm (\perm,\len,\addr) &=
                                     \begin{cases}
                                       (\perm,\len,\addr) & \text{if $\perm\in\{\exec{u},\exec{p}\}$} \\
                                       (\exec{\var{m}},\len,\addr) & \text{if $\perm = \entry{\var{m}}$}
                                     \end{cases} \\
  \nonZero{w} &=
                \begin{cases}
                  \true & \text{if $w\in \Caps$ or $w\in \nats$ and $w \neq 0$}\\
                  \false & \text{otherwise}
                \end{cases}
\end{align*}
%\Phi.reg(rv) to some other notation. It should only look up reg, if it is a regname otherwise just the litteral.
\begin{align*}
  \sem{\fail}(\Phi)     & = \failed \\
  \sem{\halt}(\Phi)     & = \halted \\
  \sem{\jmp{\lv}}(\Phi) & = \begin{cases}
                            (\Phi.\reg\update{\pcreg}{\updatePcPerm(c)}) & \text{if }\Phi.reg(lv) = c \\
                                                                         & \text{  and }c=(\perm,\len,\addr)\\
                                                                         & \text{  and }\executeAllowed(\perm)\\
                            \failed & \text{otherwise }
                            \end{cases} \\
  \sem{\jnz{\lv}{\rv}}(\Phi) & = \begin{cases}
                            \Phi.\reg\update{\pcreg}{\updatePcPerm(\var{c})} &
                            \begin{array}{l}
                              \text{if $\nonZero{\Phi.\reg(\rv)}$} \\ 
                              \text{  and $\Phi.reg(lv) = c$} \\
                              \text{  and $c=(\perm,\len,\addr)$}\\
                              \text{  and $\executeAllowed(\perm)$}
                            \end{array}
                            \\ %TODO Maybe combine with jump. (failed + this)
                            \Phi.\reg\update{\pcreg}{\Phi.\reg(\pcreg) + 1} & \text{if not $\nonZero{\Phi.\reg(\rv)}$}\\
                            \failed & \text{otherwise }
                            \end{cases} \\
 \sem{\load{\refreg{r_1}}{\refheap{r_2}}} & =
                                 \begin{cases}
                                   \Phi.\reg\update{r_1}{\var{w}} &
                                   \begin{array}{l}
                                     \text{if }\Phi.\reg(r_2) = (\perm,\len,\addr)\\
                                     \text{  and }\readAllowed{\perm} \text{ and } \var{w} = \Phi.\heap(\addr)
                                   \end{array}\\
                                   \failed & \text{otherwise }
                                 \end{cases}\\
 \sem{\store{\refheap{r_1}}{\refreg{r_2}}} & =
                                 \begin{cases}
                                   \Phi.\heap\update{\addr}{\var{w}} &
                                   \begin{array}{l}
                                     \text{if }\Phi.\reg(r_1) = (\perm,\len,\addr)\\
                                     \text{  and }\writeAllowed{\perm} \text{ and } \var{w} = \Phi.\reg(r_2)
                                   \end{array}\\
                                   \failed & \text{otherwise }
                                 \end{cases}\\
 \sem{\move{\refreg{r_1}}{\rv}} & =
                                 \begin{cases}
                                   \Phi.\reg\update{r_1}{\Phi.\reg(\rv)} & \text{if $r_1 \neq \pcreg$} \\
                                   \failed   & \text{otherwise }
                                 \end{cases}\\
\end{align*}

\section{Related reading}
\label{sec:related-reading}

This is a list of related work that might be interesting to read in the context
of this project.

\subsection{Capability machines}
\label{sec:rw-cap-machines}

\subsubsection{M-Machine}
More than 20 years ago, \cite{Carter:1994:HSF:195473.195579} have described the
use of capabilities in the M-Machine. They do seem to have a reference for the
instruction set after all~\citep{Dally1997Memo59}; it seems like the server was
just temporarily down when we were looking for this the first time...

\subsubsection{CHERI}

The CHERI processor is a much more recent capability machine, described
by~\cite{Woodruff:2014:CCM:2665671.2665740,Watson2015Cheri}.

Another result of this project is also CheriBSD: an adaptation of FreeBSD to the
CHERI
processor.\footnote{\url{http://www.cl.cam.ac.uk/research/security/ctsrd/cheri/cheribsd.html}}
It is not separately described in a published paper, but mentioned in the papers
cited above and in some tech reports (see url). This work includes a
pure-capability ABI that could provide some interesting examples.

The CHERI team also has a webpage with all of their CHERI-related publications
(including TRs and
such)\footnote{\url{http://www.cl.cam.ac.uk/research/security/ctsrd/cheri/}}.

\subsection{Logical Relations}
\label{sec:rw-log-rel}

Some papers on logical relations that are relevant for this work are the
following:

\cite{Hur:2011:KLR:1926385.1926402} describe a logical relation between ML and
a (standard) assembly language for expressing compiler correctness.  Relevant
because they target an assembly language, and they use biorthogonality.

\cite{Dreyer:2010:IHS:1863543.1863566} describe a logical relation for a ML-like
language and use public/private transitions to reason about well-bracketed
control flow. Relevant because we are considering to cover an example of
enforcing well-bracketed control flow in a capability machine.

\cite{Devriese:2016ObjCap} describe a logical relation for a JavaScript-like
language with object capabilities.  Relevant because it treats object
capabilities, albeit in a JavaScript-like lambda calculus.

\bibliographystyle{plainnat}
\bibliography{refs}

\end{document}
