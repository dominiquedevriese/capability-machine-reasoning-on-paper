%\todo[inline]{Write small introduction. Maybe something about the difficulty of full-abstraction proofs (with references?) but mention that this is easier than some other cases (reference approximate back translation?) because back translation is trivial.}
% \begin{itemize}
% \item Logical relation
% \item FTLR
% \item Sketch high-level structure of the proof
% \end{itemize}
To prove Theorem~\ref{thm:full-abstraction}, we will essentially show that trusted components in \srccm{} are related in a certain way to their embeddings in \trgcm{}, and that untrusted \trgcm{} components are similarly related to their embeddings in \srccm{}.
We will then prove that these relations imply that the combined programs have the same observable behavior, i.e.\ one terminates if and only if the other does.
The difficult part is to define when components are related.
In the next section, we give an informal overview of the relation we define, and then we sketch the full-abstraction proof in Section~\ref{subsec:proof-sketch}.

\subsection{Kripke worlds}
\label{subsec:worlds}
The relation between \srccm{} and \trgcm{} components is non-trivial: essentially, we will say that components are related if invoking them with related values produces related observable behavior.
However, values are often only related under certain assumptions about the rest of the system.
For example, the linear data part of a return capability should only be related to the corresponding \srccm{} capability if no other value in the system references the same inactive stack frame and it is sealed with a seal only used for return pointers to the same code location.
To accommodate such conditional relatedness, we construct the relation as a step-indexed Kripke logical relation with recursive worlds.
\begin{jversion}
In this presentation, we try to include most of the details.
We refer to the conference version of this paper for a presentation\lau{insert reference} with fewer details and emphasis on the intuition.
\end{jversion}

\begin{jversion}
Assumptions about the system that relatedness is predicated on are gathered in (Kripke) worlds.
A world is a semantic model of the memory.
In its simplest form, it is a collection of invariants that the memory must satisfy.
The invariants of a world can vary in complexity and expressiveness depending on the application.
The possible contents of the memory also influences what the world looks like.
For instance, the uniqueness of the linear capabilities on \trgcm{} and \srccm{} are modelled by the worlds.
In order to related \trgcm{} and \srccm{}, we need to model all the features of \srccm{} in the worlds which means we have to model:
\begin{itemize}
\item Three kinds of memory: heap, stack of local frames, and free stack
\item Linearity
\item Call stack
\item Seals
\end{itemize}

\subsubsection{Triple world and regions}
\srccm{}s memory is split in three: heap, free stack and encapsulated local stack frames.
In order to model the three kinds of memory, we simply have three sub-worlds.
That is, our world is defined as
\[
  \World = \Worldh \times \Worlds \times \Worldfs
\]
The sub-worlds are partial maps from names $\RegName$, modelled as natural numbers, to invariants.
In order to define what this actually means, we need to be more precise about what we mean by invariants.
We call the invariants regions because, as we will see later, they turn out to not be invariant.
A region describes a collection of related memory segments, so it is simply represented as a relation over memory segments ($\Rel{\MemSeg \times \MemSeg}$).
Intuitively, we want to be able to say that two memory segments are related when their contents are related.
That is, for every address in the two memory segments, the words that resides there must be related.
In Section~\ref{subsec:logical-relation}, we define precisely what it means for words to be related, but for now consider the intuition.
Say we have two integers, then they are related if they are equal.
If we instead have two capabilities, then they should also be related if they in some sense are equal, but what does it mean for capabilities to be ``equal''?
Intuitively, it should mean that the capabilities give you the same authority, for instance, if you have two executable capabilities, they should observably do the same computation.
As a capability points to a piece of memory, the authority of the capability depends on the contents of that memory.
To say whether two capabilities are related, we must know the possible contents of the memory.
This is exactly what our world express, so our regions will be world indexed, i.e.
\[
  \Worldh = \RegName \parfun (\World \fun \Rel{\MemSeg \times \MemSeg})
\]
At this point, we can see that we have constructed a recursive domain equation.
If we inline $\Worldh$ in $\World$, then we have a circular equation with no solution because the self-reference happens in a negative position.
Luckily, we can solve circular equations if we move to a different domain.
For now, we will ignore the problem and return to the issue in Section~\ref{subsubsec:rec-dom-eq}.

For the sake of readability, we introduce the following notation
\begin{gather*}
  \pwheap = \pi_1(W)\\
  \pwpriv = \pi_2(W)\\
  \pwfree = \pi_3(W)
\end{gather*}

If we just wanted to model the memory, then the above regions would suffice, but we also need the world to model linearity and seals.
In the following subsections, we will gradually introduce the necessary components for our worlds.

\subsubsection{Linearity and world joins}
% Linearity
The linear capabilities of \srccm{} and \trgcm{} guarantee that they have the sole authority over the memory they can reference.
In order to model this uniqueness, we need to keep track of which parts of memory that are uniquely referenced and make sure that only one capability references the unique parts.
%% Bookkeeping done by worlds
We use the world to keep track of what parts of memory must be uniquely referenced by having two kinds of regions: shared and spatial.
If a memory segment is governed by a shared region, then non-linear capabilities may reference it.
On the other hand, if a memory segment is governed by a spatial region, then only linear capability may reference it.
%%% spatial/spatial_owned
We cannot let multiple linear capabilities reference the same memory, so we add ownership to spatial regions.
It is only the spatial owned regions that can be referenced by a capability.
Specifically, we add tags $\spatial$ and $\spatialo$ to the spatial regions:
\[
  \Regions = \left\{
  \begin{array}{l}
    \{\spatial \} \times (\World \fun \Rel{\MemSeg \times \MemSeg}) \uplus \\
    \{\spatialo \} \times (\World \fun \Rel{\MemSeg \times \MemSeg})\uplus \\ 
  \end{array} \right.
\]
For readability, we also add a tag $\pure$ to the shared regions:
\[
  \Regionh = \{\pure \} \times (\World \fun \Rel{\MemSeg \times \MemSeg}) 
\]

% What parts of memory are linear
We will extend the regions further in Sections~\ref{subsubsec:seals}~and~\ref{subsubsec:ft-and-revocation}, but for now we have introduced enough to finish the definition of each of the three sub-worlds.
The sub-world $\Worldh$ specifies the heap memory which can be referenced by both linear and non-linear capabilities, so it should contain both shared and spatial regions.
For this reason, it is defined as
\[
  \Worldh = \RegName \parfun (\World \fun \Regionh + \Regions)
\]
\srccm{} internalizes the \stktokens{} stack, so it can only be referenced by linear capabilities which means that the two stack regions should only have spatial regions.
For instance the $\Worldfs$ is defined as
\[
  \Worldfs = \RegName \parfun (\World \fun \Regions)
\]
$\Worlds$ can also only be referenced by linear capabilities, so it should also only have spatial regions.
$\Worlds$ not only models the memory contents of the local stack frames, it also models the call stack that consists of the stack frames.
Conceptually, this means that each of the stack frames is connected with a return point in some code.
In a traditional C calling convention, the code return point would even be stored in the stack frame.
However, with \stktokens{} a stack frame does not contain any information about the corresponding code return point.
Instead, the stack frame and code return point is connected by the capabilities that reference them as they are sealed with the same seal and together constitute the sealed return pair.
In order to ensure that each call actually returns to the correct point of the code, we must still include the address of the return point in our model.
To this end, each shared region in $\Worlds$ must be paired with a return address:
\[
\Worlds = \RegName \parfun (\World \fun \Regions) \times \Addr 
\]

% Ensuring linearity
The spatial regions add the necessary bookkeeping to the worlds to model linear capabilities.
The logical relation presented in Section~\ref{subsec:logical-relation} uses this bookkeeping to ensure that linear capabilities uniquely references part of memory.

\subsubsection{Seals}
\label{subsubsec:seals}
% StkTokens designate seals for certain purposes, return, closure, etc
In order to guarantee well-bracketed control flow and local-state encapsulation, \stktokens{} designates seals for return capabilities and closure seals and relies on designated seals only being used for their intended purpose.
The designation of return seal is a system invariant that is modelled in the world as a \textit{seal interpretation function}.
This function takes a seal and returns a world-indexed relation that relates all the sealables that may be sealed with this specific seals.
That is
\[
  \Seal \parfun \World \fun \Rel{\SealableCaps \times \SealableCaps}
\]
%% Shared region has a seal indexed invariant that specifies what sealables can be sealed with a specific seal
In Section~\ref{subsubsec:ft-and-revocation}, we will see that not all regions are invariant.
The seal interpretation function should be an invariant, so we need to add it to a type of region that is invariant.
As we will see, shared regions are invariant, so we add the seal interpretation function as follows
\begin{multline*}
  \Regionh = 
  \{\pure \} \times (\World \fun \Rel{\MemSeg\times\MemSeg}) \times \\
  (\Seal \parfun \World \fun \Rel{\SealableCaps \times \SealableCaps})
\end{multline*}

\lau{Maybe add a bit more of explanation? Invariant needed for the LR to make sure that designated seals are used for their purpose. }

\subsubsection{Future worlds and revocation}
\label{subsubsec:ft-and-revocation}
% Memory evolves over time, we need to model this.
The world specifies the contents of the memory.
When a machine executes, the contents of the memory changes and parts of memory may be repurposed.
The world must model these changes.
% Future World
To this end, we have a future world relation $\future$ that describes how worlds may change over time.
For our purpose, relating \trgcm{} to \srccm{}, we are mainly interested in changes to the stack as it is the part that will be repurposed.
Specifically, we need to introduce regions to govern new encapsulated stack frames, and we need to revoke regions that govern call frames that have been popped from the call stack, so the stack frame can be reused for a new call frame.
We allow introduction of new regions simply by extension.
That is, a future world can have new regions that was not present in a past world.
Things are a bit more complicated when it comes to revocation.
\lau{Can we give a more precise explanation of why regions cannot be removed from the world (still in terms of intuition).}
For technical reasons, we need the worlds to be extensional which means that when it comes to removing regions in future worlds, we cannot simply forget about them.
Instead of removing the regions, we keep them around and tag them as removed with a $\revoked$ tag.
This means that the regions should be able to get the $\revoked$ tag which we express in a future region relation.
It is not all regions that need to be able to be revoked.
Specifically, it should only be the regions that specify memory governed by linear capabilities as linear capabilities in some sense can be revoked.
A linear capability is the sole authority over some memory, so when a program possesses a linear capability, it knows for sure that no one else has access to that memory.
This means that it is safe to repurpose that part of memory.
On the other hand, non-linear capabilities may have an alias, so it is not safe to repurpose the memory a non-linear capability has authority over.
It is only spatial regions that specify memory governed by linear capabilities, so we only add the revoked tag to the $\Regions$.
The $\revoked$ region does not specify any memory, so we drop the world indexed relation.
That is, we have
\[
  \Regions = \left\{
  \begin{array}{l}
    \{\spatial \} \times (\World \fun \Rel{\MemSeg \times \MemSeg}) \uplus \\
    \{\spatialo \} \times (\World \fun \Rel{\MemSeg \times \MemSeg}) \uplus \\
    \{\revoked \}
  \end{array} \right.
\]
The rules for the future region relation are displayed in Figure~\ref{fig:ft-reg-rel}.
% Revocation of spatial regions. 'revocation' of linear capabilities.
All regions are future regions of themselves, so regions are allowed to stay unchanged.
The $\spatial$ regions can become $\revoked$ which means that nothing can depends on the region anymore.
% Spatial -> spatialo ~ affine ?
\todo[inline]{Lau: explain why $\spatialo \future \spatial$ gives us affine capabilitites. This is claimed in the technical report, but I don't see how this is different from revoking $\spatial$ regions and adding a new region with the future world relation.}
\begin{figure}[htb]
  \centering
  \begin{mathpar}
  \inferrule{ }{ \revoked \future (\spatial,\_)}
  \and
  \inferrule{ }{ (\spatialo,H) \future (\spatial,H)}
  \and
  \inferrule{ r \in \Regions \cup \Regionh }{ r \future r}
\end{mathpar}
  \caption{Future region relation.}
  \label{fig:ft-reg-rel}
\end{figure}

With the future region relation in place, we define the future world relation as follows:
\begin{mathpar}
  \inferrule{ \text{for $i \in \{\mathrm{heap},\mathrm{free},\mathrm{priv} \}$} \\ \exists m_i : \RegionName \fun \RegionName, \text{ injective}\ldotp \dom(W'.i) \supseteq \dom(m_i(W.i)) \wedge \forall r \in \dom(W.i)\ldotp W'.i(m_i(r)) \future W.i(r) }
            { W' \future W }
\end{mathpar}
The relation says that each of the three worlds must be an extension of the past world and for each of the existing regions that must be a future region.
% injective function
Note that the future world relation has a mapping function $m_i$ which allows us to change the naming of regions in future worlds.
The definition is a generalization of the standard definition where $m_i$ would be the identity\footnote{In \citet{Skorstengaard:esop18} the future region relation and the reasoning about the awkward example could have been simplified with this future world relation.}.

% TODO monotonicity of regions

\subsubsection{Solving the recursive domain equation}
\label{subsubsec:rec-dom-eq}
%solving the recursive domain equation means that we can present the actual definitions
In the previous sections, we have sketched what we would like our worlds to look like.
However, the worlds we want constitute a self-referential domain equation for which no solution exists in set theory.
% Move to a different domain
Therefore, we need to move to a different domain with enough structure for such an equation to have a solution.
Specifically, we move to a setting where instead of sets we have c.o.f.e.'s (complete ordered families of equivalences), instead of functions we have non-expansive functions, and instead of relations we have downwards-closed relations.
Intuitively, a c.o.f.e. is a set with some extra structure.
Specifically, c.o.f.e.'s have a step-index which provide sufficient structure for us to solve recursive domain equations.
In this section we present the relevant definitions, the final world domain equation, and sketch what needs to be proven to get a solution to the recursive domain equation.

% Cofe's
In the following, we present complete ordered family of equivalences \citep{di_gianantonio_2002}.
We loosely follow the presentation in \citet{Birkedal:tutorial-notes}.
\begin{definition}[Ordered family of equivalences (o.f.e.)]
An \emph{ordered family of equivalences} is a pair $\cofe{X}$ where $X$ is a set and for each $n$ $\nequal$ is an equivalence relation.
The pair must satisfy the following properties
\begin{enumerate}
\item $\nequal[0]$ is the total relation.
\item For all $n \in \nats$, $\nequal[n] \supseteq \nequal[n+1]$.
\item For all $x,y \in X$,  if for all $n \in \nats$ $x \nequal y$, then $x = y$.
\end{enumerate}
\end{definition}
\noindent One can think of the indexes on the family of equivalences as a measure of precision.
The larger $n$ is, the more refined the equivalence may be.
On the other hand as $n$ decreases, the equivalence becomes more and more imprecise and may be unable to distinguish elements that where distinguishable at higher indices.
At index 0, all precision is lost and the equivalence cannot distinguish anything.
% TODO: Lau write intuition for sequences and why they are necessary.
\begin{definition}[Complete ordered family of equivalences (c.o.f.e.)]
  \label{def:cauchy-sequence}
  A \emph{complete ordered family of equivalences} is an ordered family of equivalences
  $\cofe{X}$ such that every Cauchy sequence in $X$ has a limit
  in $X$.

  Let $\cofe{X}$ be an o.f.e. and $\seq{x}$ be a sequence of
  elements of $X$. Then $\seq{x}$ is a \emph{Cauchy sequence} if
  \begin{align*}
    \forall k \in \nats, \exists j \in \nats, \forall n \geq j, x_j \nequal[k] x_n
  \end{align*}
  or in words, the elements of the chain get arbitrarily close.

  An element $x \in X$ is the \emph{limit} of the sequence $\seq{x}$ if
  \begin{align*}
    \forall k \in \nats, \exists j \in \nats, \forall n \geq j, x \nequal[k] x_n.
  \end{align*}
\end{definition}
Functions between c.o.f.e.'s must retain the added structure relatively to sets.
To this end, we require all our functions to be \emph{non-expansive} which means that $n$-equivalences are preserved.
A function is called \emph{contractive} when it not only retain equivalences but makes them more precise.
\begin{definition}
  \label{def:nonexpansive-contractive-ofe}
  Let $\left(X,\left(\nequal[n]_X\right)_{n=0}^{\infty} \right)$ and $\left(Y,\left(\nequal[n]_Y\right)_{n=0}^{\infty} \right)$ be two ordered families of equivalences and $f$ a function from the set $X$ to the set $Y$.
  We 
  The function $f$ is
  \begin{description}
  \item[non-expansive] when for all $x, x' \in X$, and all $n \in \nats$,
\[
  x \nequal_X x' \implies f(x) \nequal_Y f(x')
\]
  \item[contractive] when for any $x, x' \in X$, and any $n \in \nats$,
\[
  x \nequal_X x' \implies f(x) \nequal[n+1]_Y f(x') 
  \qedhere
\]
  \end{description}
\end{definition}
\noindent The step-indexing provided by c.o.f.e.'s give enough structure to solve the recursive domain equation.
However, this is not the only structure we would like to impose on our sets.
In Section~\ref{subsubsec:ft-and-revocation}, we described a future world relation which models changes to the memory.
The future world relations is essentially a preorder structure that we would like to impose on our sets.
We impose this structure on our sets by moving to \emph{preordered c.o.f.e.'s}.
% preordered cofe
\begin{definition}[Preordered c.o.f.e.]
  A preordered c.o.f.e.\ is a c.o.f.e.\ equipped with a preorder on $X$, $\left(X, \left(\nequal\right)_{n=0}^{\infty}, \future \right)$. 
  \begin{itemize}
  \item The ordering preserves limits. That is, for Cauchy sequences $\{a_n\}_n$ and $\{b_n\}_n$ in $X$ if $\{a_n\}_n \future\{b_n\}_n$, then $\lim \{a_n\}_n \future \lim \{b_n\}_n$.
  \end{itemize}
\end{definition}
The new structure we add to the set, the preorder, must be consistent with the existing structure which is why it must preserve limits.

Functions between preordered c.o.f.e's must preserve the ordering which means that they must be monotone with respect to the ordering.
% monotone functions
\begin{lemma}
  \label{def:monotone-preordered-cofe}
  Let $\left(X,\left(\nequal[n]_X\right)_{n=0}^{\infty},\future_X \right)$ and $\left(Y,\left(\nequal[n]_Y\right)_{n=0}^{\infty} ,\future_Y\right)$ be two preordered c.o.f.e.'s and $f$ a function from the set $X$ to the set $Y$.
  We 
  The function $f$ is
  \begin{description}
  \item[monotone] when for all $x, x' \in X$,
\[
  x' \future_X x \implies f(x) \future_Y f(x')
\]
  \end{description}
\end{lemma}
% Not sure this is necessary:
% % preordered cofe constructions
% %% cofe to preordered cofe
% \begin{lemma}[c.o.f.e.'s are preordered c.o.f.e's]
%   A c.o.f.e $\cofe{X}$, is also a preordered c.o.f.e. with the order $=$.
% \end{lemma}
In order to express the domain equation, we must lift all parts of the world definition from sets to c.o.f.e.'s.
That is, we need to lift the sets of partial functions, monotone non-expansive functions, and relations, respectively to c.o.f.e.'s.

We need to lift the set of relations over $R$, $\Rel{R}$, to a c.o.f.e.
However, the set of relations have no inherent step-index which means that we need to add one.
We need to add the step-index in such a way that we can define a family of equivalences that satisfies the c.o.f.e.\ conditions.
To this end, we define the uniform relations over $R$ $\URel{R}$.
%% URel
\begin{definition}[Uniform relation]
  Given a relation $R \subseteq X \times Y$, we define $\URel{R} \subset \powerset{\nats \times R}$ as
  \[
    \URel{R} = \{ A \in \powerset{\nats \times R} \mid \forall n \in \nats\ldotp \forall r \in R \ldotp \npair{r} \in A \implies \forall m \leq n \ldotp (n,r) \in A\}
  \]
  The c.o.f.e.\ $\cofe{\URel{R}}$. Define erasure as
  \[
    \erase{A}{n} \defeq \{(k,a) \in A \mid k < n\}
  \]
  and the family of equivalences as
  \[
    A \nequal B \text{ iff } \erase{A}{n} = \erase{B}{n}
  \]
\end{definition}
The $n$-equivalences of the uniform relation basically forgets everything from the step and up and requires the rest to be equal.
This means that everything is equal at step $0$ as everything is left out.

Next, we define the preordered c.o.f.e.\ of monotone and non-expansive functions.
%% Monotone, non-expansive
\begin{lemma}[Preordered c.o.f.e. of monotone and non-expansive functions]
  Given preordered c.o.f.e.\ $\left(W,\left(\nequal[n]_W\right)_{n=0}^{\infty}, \future_W\right)$ and a c.o.f.e.\ $\left(U,\left(\nequal[n]_U\right)_{n=0}^{\infty}\right)$, define the preordered c.o.f.e.\ $\left(W \monnefun U,\left(\nequal[n]\right)_{n=0}^{\infty}, \future \right)$ as follows 
  \[
    h \nequal h' \text{ iff } \forall w \in \dom(h) \ldotp h(w) \nequal_U h'(w)
  \]
  and preorder
  \[
    h \future h' \text{ iff } \forall r \in \dom(h) \ldotp h(w) = h'(w)
  \]
\end{lemma}
Next, we define the product preordered c.o.f.e.
%% Product
\begin{lemma}[Product preordered c.o.f.e.]
  Two preordered c.o.f.e's $\left(X, \left(\nequal_X\right)_{n=0}^{\infty}, \future_X \right)$ and $\left(Y, \left(\nequal_Y\right)_{n=0}^{\infty}, \future_Y \right)$ can be composed to a product preordered c.o.f.e $\left(X \times Y, \left(\nequal\right)_{n=0}^{\infty}, \future \right)$
  where the family of equivalences are defined as follows
  \[
    (x,y) \nequal (x',y') \text{ iff } x \nequal_X x' \wedge y \nequal_Y y'
  \]
  and the order as follows
  \[
    (x,y) \future (x',y') \text{ iff } x \future_X x' \wedge y \future_Y y'
  \]
\end{lemma}
We use the product preordered c.o.f.e.\ when we just need to compose two existing preordered c.o.f.e.'s into a new preordered c.o.f.e.
However, in the case of our regions, this do not suffice as they need to use the future region relation we defined in Section~\ref{subsubsec:ft-and-revocation}.
We therefore define separate preordered c.o.f.e.'s that we use to define these.
\begin{lemma}[$\Regions$ preordered c.o.f.e.]
  Given preordered c.o.f.e.\ $\left(H, \left(\nequal_H\right)_{n=0}^{\infty}, \future_H \right)$ define $\left(\{\spatial\} \times H + \{\spatialo\} \times H + \revoked, \left(\nequal\right)_{n=0}^{\infty}, \future \right)$
  where $\future$ is the future region relation from Section~\ref{subsubsec:ft-and-revocation} and
  \[
    r \nequal r' \text{ iff } n=0 \vee \left\{
      \begin{aligned}
      (\_,h),(\_,h') \in \{ \spatial  \} \times H \wedge h \nequal h' \vee \\
      (\_,h),(\_,h') \in \{ \spatialo \} \times H \wedge h \nequal h' \vee \\
      r,r' = \{revoked\}      
    \end{aligned}\right.
  \]
\end{lemma}

%% Sum
Sum c.o.f.e.'s are defined in a similar manner.

\todo[inline]{Lau: Do we need to use the future world relation here? Generally where do we need to use the prior defined relations?}
%% Partial function from set
\begin{lemma}[Preordered c.o.f.e.\ of partial functions]
  Given a set $S$ and a preorder c.o.f.e $P$, the set of partial functions $S \parfun P$ can be lifted to a preordered c.o.f.e.\ with family of equivalences
  \[
    f \nequal[0] g \wedge f \nequal g \text{ iff } \forall x \in \dom(f) \ldotp f(x) \nequal g(x)
  \]
  and preorder
  \[
    f \future g \text{ iff } \forall x \in \dom(f) \ldotp f(x) \future g(x)
  \]
\end{lemma}


%% Later cofe
\begin{lemma}[$\blater$ preordered c.o.f.e.]
  Given a preordered c.o.f.e $\left(X, \left(\nequal_X\right)_{n=0}^{\infty}, \future_X \right)$ define
  \[
    \blater\left(X, \left(\nequal\right)_{n=0}^{\infty}, \future \right) = \left(X, \left(\nequal[n]'\right)_{n=0}^{\infty}, \future \right)
  \]
  where
  \[
    x \nequal[0]' x' \wedge x \nequal[n]' x' \texttt{ iff } x \nequal[n+1] x'
  \]
\end{lemma}

% partial functions
% 


% New world definitions
The world definitions we presented so far needs to move to the domain of preordered c.o.f.e's. Morally, the definitions are the same, though. The definition of the two kinds of regions now uses $\URel{}$ rather than $\Rel$, and the world indexed-functions are monotone. Finally, all functions are now non-expansive.
\begin{multline*}
  \Regionh = 
  \{\pure \} \times (\Wor \monnefun \URel{\MemSeg^2}) \times \\
  (\Seal \parfun \Wor \monnefun \URel{\SealableCaps \times \SealableCaps})
\end{multline*}
and
\[
  \Regions = \left\{
  \begin{array}{l}
    \{\spatial \} \times (\Wor \monnefun \URel{\MemSeg^2}) \uplus \\
    \{\spatialo \} \times (\Wor \monnefun \URel{\MemSeg^2})\uplus \\ 
    \{\revoked\}
  \end{array} \right.
\]


% Reference to solution?
%% Aleš note? Tutorial note?
%% The category-theoretic solution of recursive metric-space equations, Birkedal et al?


% Argue, informally, that it is okay

% Solution theorem

\subsubsection{Joining worlds}

In order to express this, we will need to split the ownership of the world in disjoint parts.
%% Disjoint union \oplus
To this end, we introduce the operator $\oplus$ which is a disjoint union.
\begin{definition}[$\oplus$, disjoint union of ownership]
  \[
    W_1 \oplus W_2 = W
  \text{ iff }
  \begin{array}[t]{l}
    \dom(\pwheap) = \dom(\pwheap[W_1]) = \dom(\pwheap[W_2]) \tand \\
    \dom(\pwfree) = \dom(\pwfree[W_1]) = \dom(\pwfree[W_2]) \tand \\
    \dom(\pwpriv) = \dom(\pwpriv[W_1]) = \dom(\pwpriv[W_2]) \tand \\
    \forall r \in \dom(\pwheap) \ldotp \pwheap(r) = \pwheap[W_1](r) \oplus \pwheap[W_2](r) \tand \\
    \forall r \in \dom(\pwfree) \ldotp \pwfree(r) = \pwfree[W_1](r) \oplus \pwfree[W_2](r) \tand \\
    \forall r \in \dom(\pwpriv) \ldotp \pi_1(\pwpriv(r)) = \pi_1(\pwpriv[W_1](r)) \oplus \pi_1(\pwpriv[W_2](r))
  \end{array}
  \]
\end{definition}

\begin{definition}[World disjoint union $\uplus$]
  Given worlds $W_1$, $W_2$, $W$
  \[
    W_1 \uplus W_2 = W
    \text{ iff }
    \begin{array}[t]{l}
      \dom(\pwheap) = \dom(\pwheap[W_1]) \uplus \dom(\pwheap[W_2]) \tand \\
      \dom(\pwfree) = \dom(\pwfree[W_1]) \uplus \dom(\pwfree[W_2]) \tand \\
      \dom(\pwpriv) = \dom(\pwpriv[W_1]) \uplus \dom(\pwpriv[W_2]) \\
    \end{array}
  \]
\end{definition}

The two operators $\uplus$ and $\oplus$ are quite different.
The difference is most clear in the treatment of pure regions: $\uplus$ allows both worlds to have the same pure region, while $\oplus$ forbids this.
To understand this different treatment ($W_1 \uplus W_2$ and $W_1 \oplus W_2$), you should understand that the two are intended for different usages of worlds.
The $W_1 \oplus W_2$ operator treats the worlds as specifications of authority: taking the disjoint union of worlds specifying non-exclusive ownership of a block of memory is allowed and produces a new world that also specifies non-exclusive ownership of world. 
The $W_1 \uplus W_2$  operator treats worlds as specifications of memory contents: taking the disjoint union of worlds specifying the presence of the same memory range is not allowed.
The latter operator is used in the logical relation for components which specifies (among other things) that the world should specify the presence of the component's data memory.
Linking two components then produces a new component with both components' data memory.
The linked component is then valid in a world that has the combined memory presence specifications, not the combined authority.
In other words, $\oplus$ specifies disjoint authority distribution, while $\uplus$ specifies disjoint memory allocation.

Note also that this picture is further complicated by our usage of non-authority-carrying $\spatial$ regions.
They are really only there in a world $W$ as a shadow copy of a $\spatialo$ region in another world $W'$ that $W$ will be combined with.
The shadow copy is used for specifying when a memory satisfies a world: the memory should contain all memory ranges that anyone has authority over, not just the ones whose authority belongs to the memory itself.
For example, if a register contains a linear pointer to a range of memory, then the register file will be valid in a world where the corresponding region is $\spatialo$, while the memory will be valid in a world with the corresponding region only $\spatial$.
However, for the memory to satisfy the world, the block of memory needs to be there, i.e. the memory should contain blocks of memory satisfying every region that is $\spatialo$, $\pure$, but also just $\spatial$ (because it may be $\spatialo$ in, for example, the register file's world).

%% This both makes world a specification of authority as well as a specification of memory.

\subsubsection{Memory satisfaction}
\label{subsubsec:mem-sat}
% Well-formedness of world
% specification satisfaction
\[
  \memSat{\ms_S,\ms_\stk,\stk,\ms_T}{W} \text{ iff } 
  \left\{
    \begin{array}{l}
      \stk = (\opc_0,\ms_0):: \dots :: (\opc_m,\ms_m) \wedge \\
      \ms_S \uplus \ms_\stk \uplus \ms_0 \uplus \dots \uplus \ms_m  \text{ is defined} \wedge\\
      W = W_{\var{stack}} \oplus W_{\var{free\_stack}} \oplus W_{\var{heap}} \wedge\\
      \exists \ms_\var{T,stack}, \ms_\var{T,free\_stack}, \ms_\var{T,heap}, \ms_{T,f}, \ms_{S,f}, \ms_S',\overline{\sigma}\ldotp \\
      \quad \ms_S = \ms_{f,S} \uplus \ms_S' \wedge \\
      \quad \ms_T = \ms_\var{T,stack} \uplus \ms_\var{T,free\_stack} \uplus \ms_\var{T,heap} \uplus \ms_{T,f} \wedge \\
      \quad \dom(\ms_{\var{T,stack}} \uplus \var{T,free\_stack}) = [\baddr_\stk,\eaddr_\stk] \wedge \\
      \quad \{\baddr_\stk -1,\eaddr_\stk + 1\} \in \dom(\ms_{T,f}) \wedge \\
      \quad \memSatStack{\stk,\ms_\var{T,stack}}{W_{\var{stack}}} \wedge \\
      \quad \memSatFStack{\ms_\stk,\ms_\var{T,free\_stack}}{W_{\var{free\_stack}}} \wedge \\
      \quad \npair{(\overline{\sigma},\ms_S',\ms_\var{T,heap})} \in \lrheap(\pwheap)(W_{\var{heap}})
    \end{array}
  \right.
\]

\[
  \memSatStack{\stk,\ms_T}{W} \text{ iff } 
  \left\{
    \begin{array}{l}
      W_\var{stack} = \pwpriv \wedge \\
      \stk = (\opc_0,\ms_0), \dots (\opc_m,\ms_m) \wedge \\
      \forall i \in \{0,\dots,m\} \ldotp (\dom(\ms_i) \neq \emptyset \wedge\\
      \quad \forall i < j \ldotp \forall a \in \dom(\ms_i) \ldotp \forall a' \in \dom(\ms_j) \ldotp \stkb < a < a') \wedge\\
      \exists R_\ms : \dom(\activeReg{W_\var{stack}}) \fun \MemSeg \times \Addr \times \MemSeg \ldotp \\
      \quad \ms_T = \biguplus_{r \in \dom(\activeReg{W_\var{stack}})} \pi_3(R_\ms(r)) \wedge \\
      \quad \ms_0 \uplus \dots \uplus \ms_m = \biguplus_{r \in \dom(\activeReg{W_\var{stack}})} \pi_1(R_\ms(r)) \wedge \\
      \quad \exists R_W : \dom(\activeReg{W_\var{stack}}) \fun \World \ldotp \\
      \qquad W = \bigoplus_{r \in \dom(\activeReg{W_\var{stack}})} R_W(r) \wedge \\
      \qquad \forall r \in \dom(\activeReg{W_\var{stack}}), n' < n\ldotp \\
      \qquad \quad \npair[n']{(\pi_1(R_\ms(r)),\pi_3(R_\ms(r))} \in W_\var{stack}(r).H \; \xi^{-1}(R_W(r)) \wedge\\
      \qquad \quad \pi_2(R_\ms(r)) = W_\var{stack}(r).\opc \wedge\\
      \qquad \quad \exists i \ldotp \opc_i = W_\var{stack}(r).\opc \wedge \ms_i = \pi_1 (R_\ms(r))
    \end{array}
  \right.
\]

\[
  \memSatFStack{ms_\stk,\ms_T}{W} \text{ iff } 
  \left\{
    \begin{array}{l}
      W_\var{stack} = \pwfree \wedge \\
      \exists R_\ms : \dom(\activeReg{W_\var{stack}}) \fun \MemSeg \times \MemSeg \wedge \\
      \quad \ms_T = \biguplus_{r \in \dom(\activeReg{W_\var{stack}})} \pi_2(R_\ms(r)) \wedge \\
      \quad \ms_\stk = \biguplus_{r \in \dom(\activeReg{W_\var{stack}})} \pi_1(R_\ms(r)) \wedge \\
      \quad \stkb \in \dom(\ms_T) \wedge \stkb \in \dom(\ms_\stk) \wedge \\
      \quad \exists R_W : \dom(\activeReg{W_\var{stack}}) \fun \World\ldotp\\
      \qquad W = \oplus_{r \in \dom(\activeReg{W_\var{stack}})} R_W(r) \wedge\\
      \qquad \forall r \in \dom(\activeReg{W_\var{stack}}),n' < n \ldotp \\
      \qquad \quad \npair[n']{R_\ms(r)} \in  W_\var{stack}(r).H \; \xi^{-1}(R_W(r))
    \end{array}
  \right.
\]

\[
  \lrheap(\pwheap)(W') = 
  \left\{
    \npair{(\overline{\sigma},ms,\ms_T)} \middle|
    \begin{array}{l}
      \exists R_\ms : \dom(\activeReg{\pwheap}) \fun \MemSeg \times \MemSeg \wedge \\
      \quad \ms_T = \biguplus_{r \in \dom(\activeReg{\pwheap})} \pi_2(R_\ms(r)) \wedge \\
      \quad \ms = \biguplus_{r \in \dom(\activeReg{\pwheap})} \pi_1(R_\ms(r)) \wedge \\
      \quad \exists R_W : \dom(\activeReg{\pwheap}) \fun \World\ldotp\\
      \qquad W' = \oplus_{r \in \dom(\activeReg{\pwheap})} R_W(r) \wedge\\
      \qquad \forall r \in \dom(\activeReg{\pwheap}), n' < n \ldotp \\
      \qquad \quad \npair[n']{R_\ms(r)} \in  \pwheap(r).H \; \xi^{-1}(R_W(r)) \wedge\\
      \exists R_\var{seal} : \dom(\activeReg{\pwheap}) \fun \powerset{\Seal} \wedge\\
      \quad \biguplus_{r \in \dom(\activeReg{\pwheap})} R_\var{seal}(r)) \subseteq \overline{\sigma} \wedge\\
      \quad \dom(\pwheap(r).H_\sigma) = R_\var{seal}(r)
    \end{array}
  \right.
\]



% TODO take a look at notation and naming in this section. What do we want to call things
\end{jversion}
We use a type of worlds tailored to our purposes.
They consist of three sub-worlds: $\Wor = \Worldh \times \Worlds \times \Worldfs$, capturing assumptions about the heap\footnote{Actually, we use the term heap to describe all memory except the stack, including, for example, code memory.}, the inactive and the active part of the stack, respectively.
Sub-worlds consist of a finite mapping from region names to regions which come in two forms (spatial and shared):
\begin{align*}
  \Worldh  &={} \RegionName \parfun (\Regions + \Regionh)\\
  \Worlds  &={} \RegionName \parfun (\Regions \times \Addr)\\
  \Worldfs &={} \RegionName \parfun \Regions
\end{align*}
Different parts of the world can contain different types of regions: heap-related assumptions can be either spatial or shared (see below), while stack-related regions must be spatial.
Additionally, regions for inactive parts of the stack additionally include an address specifying the return address for that stack frame.

Regions in $\Regionh$ specify the presence of an invariant in the system, shared with the rest of the system.
They are tagged with the syntactic token $\pure$ and may prescribe two different types of requirements:
\begin{align*}
  \Regionh &= 
             \left\{\begin{multlined}
               \{\pure \} \times (\Wor \monnefun \URel{\MemSeg^2}) \times \\
               (\Seal \parfun \Wor \monnefun \URel{\SealableCaps \times \SealableCaps})
             \end{multlined}\right.
\end{align*}
First, they may require the presence of \srccm{} and \trgcm{} memory segments satisfying a given relation in $\Wor \monnefun \URel{\MemSeg^2}$ (readers unfamiliar with Kripke and step-indexed logical relations may read $\Wor \monnefun \URel{\MemSeg^2}$ as the set of functions from $\Wor$ to relations on memory segments).
A region might, for example, require the presence of a certain list of instructions at a certain set of memory addresses in both \srccm{} and \trgcm{}.
If a memory segment is owned by a given region, then that memory segment must be disjoint from the memory segments owned by all other regions.
Second, a shared region may also contain a partial function from seals to relations on pairs of sealable capabilities: $\Seal \parfun \Wor \monnefun \URel{\SealableCaps \times \SealableCaps}$.
When the region defines such a relation for a given seal, then no other region in the world can do the same, and any value signed with that seal will be required to satisfy the registered relation.

Spatial regions are similar to shared regions, but they are tagged as $\spatial$ or $\spatialo$ and may not specify seal invariants.
Additionally, a spatial region may also be $\revoked$:
\begin{align*}
  \Regions &={}
             \begin{multlined}
               \{\spatial,\spatialo \} \times (\Wor \monnefun \URel{\MemSeg^2}) \cup{} \{\revoked\}
             \end{multlined}
\end{align*}

The difference between spatial and shared regions is related to linearity and ownership.
For example, a \trgcm{} linear capability to a piece of memory is related to its \srccm{} counterpart, but only if no other linear capability overlaps with it.
We will model such an assumption of exclusive ownership by making the relatedness rely on the presence of a $\spatialo$ region that only one value in the system may rely on.
More concretely, we will define how to combine worlds $W_1$ and $W_2$ into a combined world $W_1 \oplus W_2$, on the condition that they represent compatible assumptions: $W_1$ and $W_2$ must contain the same regions except that they must respect exclusive ownership: $\pure$ regions must be present in both worlds, but a $\spatialo$ region can only be present in one and must be $\spatial$ in the other.
When $W_1 \oplus W_2$ is defined, we say that worlds $W_1$ and $W_2$ are compatible and we refer to $W_1$ and $W_2$ as compatible partitions of the combined world.

The reason that we have $\spatial$ regions (in addition to $\spatialo$ ones) is for defining when \srccm{} and \trgcm{} memories are related.
We need them to contain suitable memory segments for all regions in the system, even for regions owned by values that live outside the memory (for example register values).
Those regions will be in the world for the memory, but only as $\spatial$, i.e.\ they may not actually be referenced from within memory, but we still require them to be backed by suitable memory contents.
We also use a relation $W_2 \future W_1$ that defines when a world $W_2$ is a future world of a world $W_1$.
Relations that hold with respect to $W_1$, will then generally continue to hold in $W_2$.
Our future world relation is fairly standard: the future world must contain all the previous world's regions, except that $\spatial$ regions are allowed to become $\spatialo$ (i.e.\ gaining ownership of a region will never break relatedness) or $\revoked$ (i.e.\ revoking spatial regions will never break relatedness).

Attentive readers may have noticed that our definition of worlds is actually cyclic: worlds in $\Wor$ contain regions in $\Regionh$, but those contain partial functions from the set $\Wor$ to something else.
Such recursive worlds are in fact common in Kripke models, and we use the method of \citet{Birkedal:2011:SKM:1926385.1926401,Birkedal_taste_2014} (essentially an advanced form of step-indexing) to construct the set $\Wor$ and rigorously resolve the circularity.
% \begin{lemma}
%   \label{thm:recursive-domain-eq}
%   There exists a c.o.f.e.\ $\Wor$ and preorder $\future$ such that $(\Wor,\future)$ is a preordered c.o.f.e.\and there exists an isomorphism $\xi$ such that
%   \[
%     \xi : \Wor \cong \blater (\Worldh \times \Worlds \times \Worldfs)
%   \]
%   and for $\hat{W},\hat{W}' \in \Wor: \quad  \hat{W'} \future \hat{W}\text{ iff }\xi(\hat{W'}) \future \xi(\hat{W})$
% \end{lemma}

In our proof, we use only a few different types of regions.
For space reasons, we do not go into their definitions (see the \citet{technical_report}), but we give a brief overview here.
First, we have the code region $\codereg{\sigrets,\sigcloss,\mscode}$.
This is a shared region that represents the assumption that memory segment
$\mscode$ is loaded in heap memory at a certain location, it is well-formed and it uses return seals $\sigrets$ and closure seals $\sigcloss$.
% Lau: I cannot parse this sentence:
It takes ownership of those seals and registers appropriate invariants on the capabilities that may be signed with them.

A second type of regions $\stdreg{A}{\pure}$ and $\stdreg{A}{\spatialo}$ governs heap or stack memory at a set of addresses $A$.
It simply requires the presence of a memory segment for those addresses, such that the memory contains values that themselves satisfy the relation between values that we will see below.
The contents of the memory is allowed to change as long as the new contents are still valid.
Finally, a third type of regions $\stareg[(\ms_S,\ms_T)]{\spatialo}$ requires the presence of two given memory segment $\ms_S$ and $\ms_T$ and does not allow them to change.
This region is used, for example, to govern inactive parts of the stack whose contents is required to remain unmodified.

\subsection{The Logical Relation}
\label{subsec:logical-relation}
Using these Kripke worlds as assumptions, we can then define when different \srccm{} and \trgcm{} entities are related: values, jump targets, memories, execution configurations, components etc.
The most important relations are summarised in the following table, where we mention the general form of the relations, what type of things they relate and extra conditions that some of them imply:\\
% latex hack stolen from https://tex.stackexchange.com/questions/78788/align-equations-over-multiple-tabular-rows
% Lau: I find it difficult to easily see what constitutes a row in this table.
% I have updated it to a table I find easier to read.
\newcolumntype{R}{>{$}r<{$}}
\newcolumntype{L}{>{$}l<{$}}
\newcolumntype{M}{R@{${}\in{}$}L}
\begin{tabular}{|M|c|p{4.8cm}|}
  \hline
  \multicolumn{2}{|c|}{General form} & Relates ... & and ...\\
  \hline
  \npair{(w_S,w_T)} & \lrv(W) & values (machine words) & safe to pass to adversarial code\\
  \npair{(w_S,w_T)} & \lrvtrusted(W) & values (machine words) & \\
  \npair{(\reg_S,\reg_T)}  &  \lrr(W) & register files & safe to pass to adversarial code\\
  \npair{(\reg_S,\reg_T)}  &  \lrrtrusted(W) & register files & \\
  \npair{\Phi_S,\Phi_T}  &  \lro & execution configurations & \\
  \npair{(w_S,w_T)}  &  \lre(W) & $\tjmp{}$ targets &\\
  \left(\arraycolsep=1pt\array{l}(w_{S,1},w_{S,2}),\\(w_{T,1},w_{T,2})\endarray\right)  &  \lrexj(W) & $\txjmp{}{}$ targets &\\
  \multicolumn{2}{|c|}{$\memSat{\ms_S,\stk,\ms_\stk,\ms_T}{W}$} & memory & satisfy the assumptions in $W$\\
  \hline
\end{tabular}\\
These relations are defined using a set of mutually recursive equations, with cyclicity resolved through another use of step-indexing.
For space reasons, we cannot show all of these definitions, but we will try to give an overview.

Note first how we have two value relations, whose definitions are sketched in Figure~\ref{fig:value-relation}.
The difference is that the untrusted value relation $\lrv(W)$ does not just express that the two values are related, but also that they are safe to pass to an untrusted adversary, i.e. they cannot be used to break LSE and WBCF.
The trusted value relation does not have the latter requirement and is a superset of the former.

Both relations trivially include numbers $(i,i)$ which are always related to themselves.
The untrusted value relation also includes stack pointers and the underlying linear capability (with the same (non-executable) permission, range of authority, and current address), as well as syntactically equal memory capabilities, seals and sealed values, all under certain conditions involving the world $W$ and the capability's properties.

Details are in the \cite{technical_report}, but roughly, for stack capabilities, the omitted condition requires that the world contains a $\spatialo$ region governing this part of the stack.
For memory capabilities $((\perm,\lin),\baddr,\eaddr,\aaddr)$, a region in the world must govern memory $[\baddr,\eaddr]$, either $\spatialo$ or $\pure$, depending on the linearity $\lin$ of the capability.
If the capability is executable ($\perm \in \{\rx,\rwx\}$), then we additionally require that the governing region is a code region and that the two capabilities are related $\mathrm{jmp}$ targets, as expressed by the relation $\lre(W)$, in any future world (see below).

Seals allocated to trusted code are related to themselves only by $\lrvtrusted(W)$, but other seals are in both value relations.
Sealed values are in both relations essentially when the sealed values satisfy the relation that was registered for the seal in a region of the world.
Additionally, when they are combined with any other pair of values related by that relation, they must be related as $\txjmp{}{}$ targets (i.e. in $\lrexj(W)$).
Finally, capabilities to code memory are related to themselves in the trusted value relation ($\lrvtrusted(W)$) when there is an appropriate code region in the world.
They are not in the untrusted value relation because the code memory contains copies of the return seals used by the code, which must not end up in the hands of an adversary.

\begin{figure}
  \centering
  \begin{align*}
  \lrv(W) ={} & \left\{ \npair{\stpair[.]{i}{i}} \;\middle|\; i \in \ints \right\}\cup \\ &
%
    \left\{
%    \begin{array}{l}
      \npair{\left(\src{\stkptr{\perm,\baddr,\eaddr,\aaddr}}, ((\perm,\linear),\baddr,\eaddr,\aaddr) \right)} \mid\dots
        % \perm \not\in \{\rx,\rwx\} \tand \\
        % % \perm = \noperm & \Rightarrow & \npair{(\linear,\baddr,\eaddr)} \in
        % % \lrp(W) \wedge \\
        % \quad\perm \in \{\ro,\rw\} \Rightarrow \npair{[\baddr,\eaddr]} \in \stackReadCond{W} \tand \\
        % \quad\perm = \rw  \Rightarrow \npair{[\baddr,\eaddr]} \in \stackWriteCond{W}
%    \end{array}
    \right\} \cup \\ &
%
    \left\{
%    \begin{array}{l}
      \npair{\left(\src{\seal{\sigma_\baddr,\sigma_\eaddr,\sigma}}, \seal{\sigma_\baddr,\sigma_\eaddr,\sigma} \right)} \mid 
      % [\sigma_\baddr,\sigma_\eaddr] \mathrel{\#} (\sigrets \cup \sigcloss) \wedge 
                       \dots 
      % \quad\forall \sigma' \in [\sigma_\baddr,\sigma_\eaddr] \ldotp \exists r \in \dom(\pwheap) \ldotp \\
      % \quad \pwheap(r) = (\pure,\_,H_\sigma) \tand H_\sigma \; \sigma' \nequal (\lrv \circ \xi)
%    \end{array}
    \right\} \cup \\ &
        \left\{
%    \begin{array}{l}
      \npair{\left(\src{\sealed{\sigma,\vsc_S}}, \sealed{\sigma,\vsc_T} \right)} \mid \dots
      % \isLinear{\src{\vsc_S}} \text{ iff } \isLinear{\vsc_T} \tand \\
      % \quad\exists r \in \dom(\pwheap), \sigrets,\sigcloss,\mscode \ldotp \pwheap(r) = (\pure,\_,H_\sigma) \tand \\
      % \qquad H_\sigma \; \sigma \nequal H^\mathrm{code,\square}_\sigma \; \sigrets \; \sigcloss \; \mscode \; \gc \; \sigma \tand \\
      % \qquad \npair[n']{\stpair[.]{\vsc_S}{\vsc_T}} \in H_\sigma \; \sigma \; \xi^{-1}(W) \text{ for all $n' < n$}\tand\\
      % \qquad (\nonLinear{\src{\vsc_S}} \Rightarrow \\
      % \qquad\quad\forall W' \future \purePart{W}, W_o, n' < n, \npair[n']{\stpair[.]{\vsc_S'}{\vsc'_T}} \in H_\sigma \; \sigma \; \xi^{-1}(W_o) \ldotp \\
      % \qquad \qquad \npair[n']{\src{\vsc_S},\src{\vsc_S'},\vsc_T,\vsc_T'} \in \lrexj(W'\oplus W_o)) \tand \dots
      % % \quad (\isLinear{\src{\vsc_S}} \Rightarrow \\
      % % \qquad\forall W' \future W, W_o, n' < n, \npair[n']{\stpair[.]{\vsc_S'}{\vsc'_T}} \in H_\sigma \; \sigma \; \xi^{-1}(W_o) \ldotp \\
      % % \qquad \quad \npair[n']{\src{\vsc_S},\src{\vsc_S'},\vsc_T,\vsc_T'} \in \lrexj(W'\oplus W_o)) \wedge \\
%    \end{array}
    \right\}\cup\\ &
%
     \left\{ \npair{\left(\src{((\perm,\lin),\baddr,\eaddr,\aaddr)}, ((\perm,\lin),\baddr,\eaddr,\aaddr)\right)} \mid \dots
    % \begin{array}{l}
    %   [b,e] \mathrel{\#} \ta \tand\\
    %   \begin{array}{r l l }
    %     % \perm = \noperm & \Rightarrow & \npair{(\lin,\baddr,\eaddr)} \in
    %     % \lrp(W) \wedge \\
    %     \perm \in \readAllowed{} &\Rightarrow& \npair{[\baddr,\eaddr]} \in \readCond{\lin,W} \wedge\\
    %     \perm \in \writeAllowed{} &\Rightarrow& \npair{[\baddr,\eaddr]} \in \writeCond{\lin,W} \wedge\\
    %     % we are excluding rwx pointers.
    %     \perm \neq \rwx \wedge \\
    %     % \perm = \rwx &\Rightarrow&
    %     % \array[t]{l}\npair{(\{\rwx,\rx\},\baddr,\eaddr)} \in \execCond{\lin,W}
    %     % \wedge \\
    %     % \npair{(\baddr,\eaddr)} \in \xReadCond{\lin,W} \endarray\\
    %     \perm = \rx &\Rightarrow& \array[t]{l}\npair{[\baddr,\eaddr]} \in \execCond{W} \wedge\\
    %     \npair{[\baddr,\eaddr]} \in \xReadCond{W} \wedge \\
    %     \lin = \normal \\ \endarray
    %   \end{array}
    % \end{array}
     \right\} \\
%  \end{array}
  \lrvtrusted(W) ={} & \lrv(W)\cup \\
%  \begin{array}[t]{l}
    &\left\{
    %\begin{array}{l}
      \npair{\left(\src{\seal{\sigma_\baddr,\sigma_\eaddr,\sigma}}, \seal{\sigma_\baddr,\sigma_\eaddr,\sigma} \right)} \mid
      % [\sigma_\baddr,\sigma_\eaddr] \subseteq(\sigrets \cup \sigcloss) \wedge 
      \dots 
    %           \exists r \in \dom(\pwheap) \ldotp \\
    %           \quad \pwheap(r) \nequal \codereg{\sigrets,\sigcloss,\mscode,\gc} \tand \dom(\mscode) \subseteq \ta \\
    %           \quad \tand [\sigma_\baddr,\sigma_\eaddr] \subseteq (\sigrets\cup\sigcloss) \tand \sigrets \subseteq \gsigrets \tand \sigcloss \subseteq \gsigcloss
    % \end{array}
    \right\} \cup \\
    & \left\{
%    \begin{array}{l}
      \npair{\left(\src{((\perm,\normal),\baddr,\eaddr,\aaddr)},((\perm,\normal),\baddr,\eaddr,\aaddr) \right)} \mid \perm \le \rx \wedge \dots
    %   \quad \perm \sqsubseteq \rx \tand 
    %    [\baddr,\eaddr] \subseteq \ta \tand 
    %    \npair{[\baddr,\eaddr]} \in \xReadCond[\square,\gc]{W} 
    % \end{array}
    \right\}
%  \end{array}
\end{align*}
\caption{Sketches of the trusted and untrusted value relation.}
\label{fig:value-relation}
\end{figure}

\begin{figure}
  \centering
\begin{align*}
  \lrrg{\trust}(W) &= \left\{ \npair{\stpair{\reg}{\reg}} \middle|
    \begin{array}{l}
    % Lau: Consider using \Wor instead of \World as we have not made a distinction between the two.
      \exists S : (\RegName \setminus \{\pcreg \})\fun \World \ldotp \\
      \quad W = \bigoplus_{r \in (\RegName\setminus (\{\pcreg \} \cup R))} S(r) \wedge \\
      \quad \forall r \in \RegName \setminus \{\pcreg \}\ldotp \npair{\stpair[.]{\src{\reg_S(r)}}{\reg_T(r)}} \in \lrvg{\trust}(S(r))
    \end{array}
            \right\}\\
  \lre(W)&= \left\{ \begin{array}{l}
    \npair{\stpair[.]{w_{c,S}}{w_{c,T}}} | \\
    \quad\forall \src{\reg_S}, \reg_T, \src{\ms_S}, \ms_T, \src{\ms_\stk}, \src{\stk}, W_\lrrs , W_\lrm \ldotp \\
    \qquad\npair{\stpair{\reg}{\reg}} \in \lrr(W_\lrrs ) \tand \memSat{\stpair[.]{\ms_S,\stk,\ms_\stk}{\ms_T}}{W_\lrm} \tand\\
    \qquad\Phi_S = \src{(\ms_S,\reg_S,\stk, \ms_\stk)} \tand \Phi_S' = \Phi_S \updReg{\pcreg}{w_{c,S}} \tand\\
                     \qquad\Phi_T = (\ms_T,\reg_T) \tand \Phi_T' = \Phi_T\updReg{\pcreg}{w_{c,T}} \tand \\
                     \qquad W \oplus W_\lrrs \oplus W_\lrm \text{ is defined }\\
    \qquad\qquad\Rightarrow\npair{\left(\Phi_S', \Phi_T' \right)}\in \lro
  \end{array}
  \right\}
\end{align*}
  \begin{align*}
  \lro[] ={}&\{ \npair{\left(\src{\Phi_S},\Phi_T\right)} \mid
    \src{\Phi_S \sterm{}} \Leftrightarrow \Phi_T \trg{\term} \}
\end{align*}
\caption{Simplified sketches of the register file relation $\lrr(W)$, the relation
  for $\com{jmp}$ targets $\lre(W)$  and the observation relation $\lro(W)$.}
\label{fig:obs-rel}
\end{figure}

In Figure~\ref{fig:obs-rel}, we show sketches of the register file relation $\lrr$, the relation between $\tjmp{}$ targets $\lre(W)$ and the observation relation $\lro$.
The trusted/untrusted register file relation simply requires that all registers except $\pcreg$ are in the corresponding value relation (in a compatible world partition).
Two execution configurations are in the observation relation $\lro$ if one terminates whenever the other does\footnote{The actual definition in the \cite{technical_report} is complicated a bit by step-indexing and the fact that we actually use two separate observation relations for left- and right-approximation.}.
The $\lre(W)$ relation then includes any two words which can be plugged into related register files and memories (in compatible worlds), to obtain execution configurations in the observation relation.

\subsection{Fundamental Theorem}
An important lemma in our proof of full abstraction of the embedding of \srccm{} into \trgcm{}, is the fundamental theorem of logical relations (FTLR).
The name indicates that it is an instance of a general pattern in logical relations proofs, but is otherwise unimportant.
\begin{lemma}[FTLR (roughly)]
  \label{thm:ftlr}
  If $[\baddr,\eaddr] \subseteq \dom(\mscode)$ and $\pwheap(r) = \codereg{\sigrets,\sigcloss,\mscode}$, and either $[\baddr,\eaddr] \subseteq \ta$ and $\mscode$ behaves reasonably (see Section~\ref{sec:well-form-reas}) or
$[\baddr,\eaddr] \mathrel{\#} \ta$,
then 
  \[
    \npair{\left(\src{((\rx,\normal),\baddr,\eaddr,\aaddr)}, ((\rx,\normal),\baddr,\eaddr,\aaddr)\right)} \in \lre(W) \qedhere
  \]
\end{lemma}

Roughly speaking, this lemma says that under certain conditions, executing any executable capability under \srccm{} and \trgcm{} semantics will produce the same observable behavior.
The conditions require that the capability points to a memory region where code is loaded and that code must be either trusted and behave reasonably (i.e.\ respect the restrictions that \stktokens{} relies on, see Section~\ref{sec:well-form-reas}) or untrusted (in which case, it cannot have WBCF or LSE expectations, see Section~\ref{sec:well-form-reas}).

The proof of the lemma consists of a big induction where each possible instruction is proven to behave the same in source and target in related memories and register files.
After that first step, the induction hypothesis is used for the rest of the execution.

\subsection{Full Abstraction Proof Sketch}
\label{subsec:proof-sketch}
Using Lemma~\ref{thm:ftlr}, we can now proceed to proving Theorem~\ref{thm:full-abstraction} (full abstraction).
First, we extend the logical relation into an omitted relation on components $(\var{comp}_S,\var{comp}_T) \in \lrcomp(W)$.
% Essentially, it relates a component to itself if instantiating their imports with related values produces related exports and code memory satisfying the appropriate code region in the world.
% The component relation $\mathcal{C}(W)$ basically lifts the logical relation we have presented above to components.
% The component relation relates a component $(\mscode,\msdata,\overline{\mathrm{import}},\overline{\mathrm{export}},\sigrets,\sigcloss)$ to itself when two conditions are satisfied.
% First, when words that relate to them selves in the untrusted value relation are used to satisfy the imports, i.e.\ the words are placed on the import addresses in $\msdata$, and this data memory is combined with the code memory $\mscode$, then it forms a safe heap, i.e.\ it is in the $\mathcal{H}$ relation.
% Second, the exports should always be safe to use which means that they must be in the untrusted value relation in any future world.
% \begin{multline*}
%   \lrcomp(W) =\\
%   \left\{\begin{aligned}
%       &\npair{\var{comp},\var{comp}} \;\mid \;\\
%       &\qquad\var{comp} = (\mscode,\msdata,\overline{a_{\mathrm{import}} \mapsfrom s_{\mathrm{import}}},\overline{s_{\mathrm{export}} \mapsto w_{\mathrm{export}}},\sigrets,\sigcloss) \tand \\
%       &\qquad\text{For all } W' \future W \ldotp \\
%       &\qquad\quad\text{If } \overline{\npair[n']{(w_{\mathrm{import}},w_{\mathrm{import}})}} \in \lrv(\purePart{W'}) \text{ for all $n' < n$}\\
%       &\qquad\quad\text{and } \msdata' = \msdata{}[\overline{a_{\mathrm{import}} \mapsto w_{\mathrm{import}}}] \\
%       &\qquad\quad\text{then } \npair{(\sigrets\uplus\sigcloss,\mscode\uplus \msdata', \mscode\uplus\msdata')} \in \lrheap(\pwheap)(W') \tand\\
%       &\qquad\quad\overline{\npair{(w_{\mathrm{export}},w_{\mathrm{export}})}} \in \lrv(\purePart{W'})
%     \end{aligned}
%   \right\}\\
% \cup \left\{
%     \begin{multlined}
%       \npair{(\var{comp}_0,c_{\mathrm{main},c}, c_{\mathrm{main},d}),(\var{comp}_0,c_{\mathrm{main},c}, c_{\mathrm{main},d})} \;\mid \;\\
%       \npair{(\var{comp}_0,\var{comp}_0)} \in \lrcomp(W) \tand
%       \{(\_ \mapsto c_{\mathrm{main},c}),(\_ \mapsto c_{\mathrm{main},d})\} \subseteq \overline{w_{\mathrm{export}}}
%     \end{multlined}
%   \right\} 
% \end{multline*}
%
Using Lemma~\ref{thm:ftlr} and the definitions of the logical relations, we can then prove the following two lemmas.
The first is a version of the FTLR for components, stating that all components are related to themselves if they are either (1) well-formed and untrusted or (2) well-formed, reasonable and trusted.
\begin{lemma}[FTLR for components]
  \label{lem:ftlr-comps}
  If $\comp$ is a well-formed component, i.e. $\wdjud{\comp}$ and either
    $\dom(\comp.\mscode) \subseteq \ta$ and $\src{\comp}$ is a reasonable component; or
    $\dom(\comp.\mscode) \mathrel{\#} \ta$,
  then there exists a $W$ such that
  $\npair{(\src{\comp},\comp)} \in \lrcomp(W)$.
\end{lemma}

Another lemma then relates the component relation and context plugging: plugging related components into related contexts produces related execution configurations.
\begin{lemma}
  \label{lem:adeq-context-plug}
  If $\npair{\stpair{\context}{\context}}\in\lrcomp(W_1)$ and $\npair{\stpair{\comp}{\comp}} \in \lrcomp(W_2)$ and $W_1\oplus W_2$ is defined, then
  $\plug{\src{\context_S}}{\src{\comp_S}}$ terminates iff $\plug{\context_T}{\comp_T}$ terminates.
\end{lemma}
% Finally, we have an adequacy lemma for the execution configuration relation.
% This lemma says that for related \srccm{} and \trgcm{} configurations, one terminates iff the other does.
% \begin{multline*}
%   \lrec[\square,\gc = (\ta,\stkb)](W) = \\
% \left\{
%   \begin{array}{l}
%      \npair{\left(
%     (\ms_S,\reg_S,\stk,\ms_\stk),
%     (\ms_T,\reg_T)\right)} \mid \\
%     \quad \exists W_M,W_R,W_\pcreg \ldotp W = W_M \oplus W_R \oplus W_\pcreg \tand\\
%     \qquad \npair{( (\reg_S(\pcreg),\reg_S(\rdata)), (\reg_T(\pcreg),\reg_T(\rdata)) )} \in \lrexj(W_\pcreg) \tand \\
%     \qquad \reg_S(\pcreg) \neq \retptrc(\_) \tand
%     \reg_S(\rdata) \neq \retptrd(\_) \tand \\
%     \qquad \nonExec{\reg_S(\rdata)} \tand  
%      \nonExec{\reg_T(\rdata)} \tand \\
%     \qquad \memSat{\ms_S,\ms_\stk,\stk,\ms_T}{W_M} \tand \npair{\stpair{\reg}{\reg}} \in \lrr(\{\rdata\})(W_R)
%   \end{array}
% \right\}
% \end{multline*}
% \begin{lemma}[Adequacy of execution configuration logical relation]
%   \label{lem:adequacy}
%   If $\npair{\stpair{\Phi}{\Phi}}\in\lrec(W)$ then $\src{\Phi_S} \sterm{}$ iff $\Phi_T\term$.
% \end{lemma}

Finally, we use these two lemmas to prove Theorem~\ref{thm:full-abstraction}.
\begin{proof}[Proof of Theorem~\ref{thm:full-abstraction}]
  The proofs of both directions are similar, so we only show the right direction.
  To show the \trgcm{} contextual equivalence, assume w.l.o.g\ a well-formed context $\trg{\context}$ such that ${\plug{\trg{\context}}{\src{\comp_1}} \term[]{}}$.
  The proof is sketched in Figure~\ref{fig:fa-proof-sketch}.
  By the statement of Theorem~\ref{thm:full-abstraction}, we may assume that the trusted components $\src{\comp_1}$ and $\src{\comp_2}$ are well-formed and reasonable.
  We prove arrow (1) in the figure by using the mentioned assumptions about $\src{\comp_1}$ and $\trg{\context}$ along with Lemma~\ref{lem:ftlr-comps} and \ref{lem:adeq-context-plug}.
  Now we know that ${\plug{\trg{\context}}{\src{\comp_1}} \sterm[]{}{}}$, so by the assumption that $\src{\comp_1}$ and $\src{\comp_2}$ are contextually equivalent on \srccm{} we get ${\plug{\trg{\context}}{\src{\comp_2}} \sterm[]{}{}}$, i.e.\ arrow (2) in the figure.
  To prove arrow (3), we again apply Lemma~\ref{lem:ftlr-comps}, \ref{lem:adeq-context-plug}; but this time, we use the assumption that $\src{\comp_2}$ is well-formed and reasonable and that $\trg{\context}$ is well-formed.
\end{proof}

\renewcommand{\comp}{C}
\begin{figure}
  \centering
  \begin{tikzpicture}[scale=0.8,every node/.style={scale=.9}]
    % \draw[help lines,yellow] (0,0) grid (10,7);
    \node at (5,4.7) { ${\src{\comp_1}\mathrel{\sconeq} \src{\comp_2}}$ };

    \node at (3.4,4) { ${\plug{\trg{\context}}{\src{\comp_1}} \sterm[]{\gc}{}}$ };
    \node at (5,4) { $\mathrel{\Rightarrow}$ };
    \node at (6.6,4) { ${\plug{\trg{\context}}{\src{\comp_2}} \sterm[]{\gc}{}}$ };

    \node at (4.35,2.8) { (1) };
    \node at (5,3.6) { (2) };
    \node at (5.65,2.8) { (3) };

    \draw[out=100,in=260,double,-implies,double equal sign distance] (4,2.6) to (4,3.4);

    \draw[out=280,in=80,double,-implies,double equal sign distance] (6,3.4) to (6,2.6);

    \node[align=center] at (8.2,3) { $ {\trg{\context}} \cong \trg{\context}$ \\
      $ {\src{\comp_2}} \cong {\src{\comp_2}}$};
    \node[align=center] at (1.8,3) { $ {\trg{\context}} \cong \trg{\context}$ \\
      $ {\src{\comp_1}} \cong {\src{\comp_1}}$};
    % \node at (9,2.7) { $e  {\src{C_1}} \cong \src{C_1}} : tau$ };
    % \node at (8.7,3.3) { $ {\trg{\context}} \cong \trg{\context} :{\emptyset},tau \ra e,{\cdots}$ };
    % \node at (.8,3) { $e  {\src{C_1}} \cong {\src{C_1}} : tau$ };

    \node at (3.4,2) { ${\plug{\trg{\context}}{\src{\comp_1}} \term[]{}}$ };
    \node at (5,2.1) { $\overset{?}{\Rightarrow}$ };
    \node at (6.6,2) { ${\plug{\trg{\context}}{\src{\comp_2}} \term[]{}}$ };

    \node at (5,1.3) { ${\src{\comp_1}}\mathrel{\overset{?}{\tconeq}}{\src{\comp_2}}$ };

    \draw[out=-90,in=90,double,-implies,double equal sign distance] (0,5) to node[sloped, yshift =.7em]{\Small Contextual equivalence preservation} (0,1);
  \end{tikzpicture}
  \caption{Proving one direction of fully abstract compilation (contextual equivalence preservation).}
  \label{fig:fa-proof-sketch}
\end{figure}


% \subsection{Proof sketch}
% \label{subsec:proof-sketch}
% \begin{proof}[Proof of Theorem~\ref{thm:full-abstraction}]
  % \item Consider first the upward arrow.
  %   Assume $\src{\var{comp}_1} \tconeq \src{\var{comp}_2}$.

  %   Take a $\src{\context}$ such that $\vdash \src{\context}$, take $\src{\ta[,i]}
  %   = \src{\dom(\var{comp}_i.\mscode)}$, $\gsigrets_i = \var{comp}_i.\sigrets$ and
  %   $\gsigcloss_i = \var{comp}_i.\sigcloss$, $\gc_i = (\ta[,i],\stkb_i,\gsigrets_i,\gsigcloss_i)$ and we will prove that
  %   $\src{\plug{\context}{\var{comp}_1} \sterm{\gc_1}} \Leftrightarrow
  %   \src{\plug{\context}{\var{comp}_2} \sterm{\gc_2}}$.

  %   By symmetry, we can assume w.l.o.g. that $\src{\plug{\context}{\var{comp}_1} \sterm{\gc_1}}$ and prove that $\src{\plug{\context}{\var{comp}_2} \sterm{\gc_2}}$.
  %   Note that this implies that $\src{\context}$ is a valid context for both $\src{\var{comp}_1}$ and $\src{\var{comp}_2}$.

  %   First, we show that also $\plug{\context}{\var{comp}_1} \trg{\term}$.
  %   Take $n$ the amount of steps in the termination of $\src{\plug{\context}{\var{comp}_1} \sterm{\gc_1}}$.
  %   It follows from Lemma~\ref{lem:ftlr-comps} that $\npair[n+1]{(\var{comp}_1,\var{comp}_1)} \in \lrcomp[\preceq,\gc_1](W_1)$ for some $W_1$ with $\dom(\pwfree) = \dom(\pwpriv) = \emptyset$.
  %   It also follows from the same Lemma~\ref{lem:ftlr-comps} that $\npair[n+1]{(\context,\context)} \in \lrcomp[\preceq,\gc_1](W_1')$ for some $W_1'$ that we can choose such that $W_1 \uplus W_1'$ is defined.
  %   Lemma~\ref{lem:compat-context-plug} then tells us that $\npair{(\plug{\context}{\var{comp}_1}, \plug{\context}{\var{comp}_1})} \in \lrec[\preceq,\gc_1](W_1\uplus W_1')$
  %   Together with $\src{\plug{\context}{\var{comp}_1} \sterm[n]{\gc_1}}$, Lemma~\ref{lem:adequacy} then tells us that $\plug{\context}{\var{comp}_1} \trg{\term}$.

  %   It follows from $\src{\var{comp}_1} \tconeq \src{\var{comp}_2}$ that also $\plug{\context}{\var{comp}_2} \trg{\term}$.

  %   It now remains to show that also $\src{\plug{\context}{\var{comp}_2} \sterm{\gc_2}}$.
  %   Take $n'$ the amount of steps in the termination of $\plug{\context}{\var{comp}_2} \trg{\term}$.
  %   It follows from Lemma~\ref{lem:ftlr-comps} that $\npair[n'+1]{(\var{comp}_2,\var{comp}_2)} \in \lrcomp[\succeq,\gc_2](W_2)$ for some $W_2$ with $\dom(\pwfree) = \dom(\pwpriv) = \emptyset$.
  %   It also follows from the same Lemma~\ref{lem:ftlr-comps} that $\npair[n'+1]{(\context,\context)} \in \lrcomp[\succeq,\gc_2](W_2')$ for some $W_2'$ that we can choose such that $W_2 \uplus W_2'$ is defined.
  %   Lemma~\ref{lem:compat-context-plug} then tells us that $\npair[n']{(\plug{\context}{\var{comp}_2}, \plug{\context}{\var{comp}_2})} \in \lrec[\succeq,\gc_2](W_2\uplus W_2')$
  %   Together with $\plug{\context}{\var{comp}_2} \trg{\term[n']}$, Lemma~\ref{lem:adequacy} then tells us that $\src{\plug{\context}{\var{comp}_2} \sterm{\gc_2}}$, concluding this direction of the proof.

%   First consider the right arrow:

%     Assume $\src{\var{comp}_1} \sconeq \src{\var{comp}_2}$. Take $\src{\ta[,i]} = \src{\dom(\var{comp}_i.\mscode)}$, $\gsigrets_i = \var{comp}_i.\sigrets$ and $\gsigcloss_i = \var{comp}_i.\sigcloss$, $\gc_i = (\ta[,i],\stkb_i,\gsigrets_i,\gsigcloss_i)$.
% %
%     Take a $\trg{\context}$ such that $\vdash \trg{\context}$ and we will prove that
%     $\trg{\plug{\context}{\var{comp}_1} \term} \Leftrightarrow
%     \trg{\plug{\context}{\var{comp}_2} \term}$.
% %
%     By symmetry, we can assume w.l.o.g. that $\trg{\plug{\context}{\var{comp}_1} \term}$ and prove that $\trg{\plug{\context}{\var{comp}_2} \term}$.
%     Note that this implies that $\trg{\context}$ is a valid context for both $\trg{\var{comp}_1}$ and $\trg{\var{comp}_2}$.
% %
%     First, we show that also $\plug{\context}{\var{comp}_1} \src{\sterm{\gc_1}}$.
%     Take $n$ the amount of steps in the termination of $\plug{\context}{\var{comp}_1} \trg{\term}$.
%     It follows from Lemma~\ref{lem:ftlr-comps} that $\npair[n+1]{(\var{comp}_1,\var{comp}_1)} \in \lrcomp[\succeq,\gc_1](W_1)$ for some $W_1$ with $\dom(\pwfree) = \dom(\pwpriv) = \emptyset$.
%     It also follows from the same Lemma~\ref{lem:ftlr-comps} that $\npair[n+1]{(\context,\context)} \in \lrcomp[\succeq,\gc_1](W_1')$ for some $W_1'$ that we can choose such that $W_1 \uplus W_1'$ is defined.
%     Lemma~\ref{lem:compat-context-plug} then tells us that $\npair{(\plug{\context}{\var{comp}_1}, \plug{\context}{\var{comp}_1})} \in \lrec[\succeq,\gc_1](W_1\uplus W_1')$
%     Together with $\plug{\context}{\var{comp}_1} \trg{\term[n]}$, Lemma~\ref{lem:adequacy} then tells us that $\plug{\context}{\var{comp}_1} \src{\sterm{\gc_1}}$.
% %
%     It follows from $\src{\var{comp}_1} \sconeq \src{\var{comp}_2}$ that also $\plug{\context}{\var{comp}_2} \src{\sterm{\gc_2}}$.
% %
%     It now remains to show that also $\plug{\context}{\var{comp}_2} \trg{\term}$.
%     Take $n'$ the amount of steps in the termination of $\plug{\context}{\var{comp}_2} \src{\sterm{\gc_2}}$.
%     It follows from Lemma~\ref{lem:ftlr-comps} that $\npair[n'+1]{(\var{comp}_2,\var{comp}_2)} \in \lrcomp[\preceq,\gc_2](W_2)$ for some $W_2$ with $\dom(\pwfree) = \dom(\pwpriv) = \emptyset$.
%     It also follows from the same Lemma~\ref{lem:ftlr-comps} that $\npair[n'+1]{(\context,\context)} \in \lrcomp[\preceq,\gc_2](W_2')$ for some $W_2'$ that we can choose such that $W_2 \uplus W_2'$ is defined.
%     Lemma~\ref{lem:compat-context-plug} then tells us that $\npair[n']{(\plug{\context}{\var{comp}_2}, \plug{\context}{\var{comp}_2})} \in \lrec[\preceq,\gc_2](W_2\uplus W_2')$
%     Together with $\plug{\context}{\var{comp}_2} \trg{\term[n']}$, Lemma~\ref{lem:adequacy} then tells us that $\plug{\context}{\var{comp}_2} \trg{\term}$, concluding the second direction of the proof.

%  The left arrow is proven in a similar manner.
% \end{proof}

%%% Local Variables:
%%% TeX-master: "paper"
%%% End: